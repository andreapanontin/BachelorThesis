\begin{center}
 \subsubsection*{Ringraziamenti}
\end{center}
Ringrazio innanzitutto la mia famiglia, che mi ha sempre supportato nel corso degli studi.
Dedico un grande ringraziamento alla prof.ssa Maria Elena Jary e al prof. Franco Vergani, le cui lezioni mi hanno fatto intravedere la bellezza della matematica e mi hanno portato a seguire questo bellissimo percorso didattico.

Ringrazio i miei compagni di corso, per avermi sopportato e supportato in questi anni, condividendo la grande passione per la matematica.
Un ringraziamento va anche a tutti i professori che, durante tutti gli scorsi tre anni, mi hanno mostrato nuovi ambiti della materia, facendomi appassionare a temi di cui non sarei stato altrimenti in grado di scoprire la bellezza.

Infine desidero ringraziare il mio relatore, prof. Davide Luigi Ferrario, il quale mi ha guidato nella stesura di questo documento, prima manifestazione di indipendenza didattica.

\newpage
\begin{center}
 \subsubsection*{Struttura dell'elaborato}
\end{center}
Il seguente elaborato vuole essere un'introduzione all'argomento della computazione quantistica.
Per questo motivo presuppone semplicemente conoscenze di algebra lineare e analisi matematica di base, presenti in ogni curriculum di matematica o fisica.
Si presuppone, inoltre, che il lettore abbia un'educazione matematica, non necessariamente fisica, quindi si dedica una grande mole di spazio per introdurre concetti basilari di meccanica quantistica, presentati in qualsiasi corso su tale argomento.
Questo non è un invito, per il lettore esperto nella materia sopracitata, ad evitare di leggere le parti relative alla meccanica quantistica.
Esse, infatti, sono riprese in ottica computazionale, con alcune variazioni rispetto alla usuale trattazione fisica.

L'opera si divide in tre capitoli, un primo il cui fine è di descrivere il funzionamento di un computer quantistico, un secondo che vuole trattare la complessità computazionale e, infine, un terzo in cui sono descritti una famiglia di algoritmi quantistici.
È chiara la forte dipendenza logica del terzo capitolo dal primo.
Il funzionamento degli algoritmi, invece, non richiede conoscenze di complessità computazionale per essere compreso.

La dipendenza dei capitoli è, quindi, la seguente:
Il primo e il secondo capitolo risultano, sostanzialmente, indipendenti (a meno dell'ultima sezione del secondo capitolo, che risulta più chiara in luce di una lettura del primo).
Il terzo capitolo, principalmente, ha come scopo quello di descrivere il funzionamento degli algoritmi, obiettivo per il quale è necessario possedere le conoscenze descritte nel primo capitolo.
Il secondo capitolo è necessario, nel terzo capitolo, solo per chiarificare la nozione di richieste computazionali, usate per descrivere la velocità di esecuzione dell'algoritmo descritto.

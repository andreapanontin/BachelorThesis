\chapter{Algoritmi quantistici}\label{chap:algorithms}
Con gli strumenti sviluppati fin'ora si possono, finalmente, studiare i primi algoritmi quantistici.
Innanzitutto verranno presentati degli algoritmi che sfruttano il comportamento quantistico del modello computazionale esibendo comportamenti impossibili in quello classico.
In seguito verranno trattati degli algoritmi per la risoluzione di importanti problemi dal punto di vista pratico.
Tali algoritmi si baseranno sulla \textit{trasformata di Fourier quantistica}, una riscrittura in termini quantistici della \textit{trasformata di Fourier discreta}, alla base di molti algoritmi efficienti in termini classici.
È notevole osservare che tutti questi algoritmi mostreranno richieste computazionali decisamente minori, a volte anche in modo esponenziale, rispetto alle migliori controparti classiche.

\section{Fenomeni quantistici}\label{sec:Quantum_phenomena}
Come introdotto nella sezione \ref{sec:paragone_cc} il modello di computazione quantistica gode di svariate peculiarità rispetto alla computazione classica.
La sezione citata si concentra sulle mancanze della prima rispetto all'ultima, in questa, invece, si mostreranno alcuni fenomeni che non sono replicabili classicamente.
In particolare si tratteranno alcuni possibili vantaggi ottenibili con il nuovo paradigma di computazione.

\subsection{Stati di Bell}\label{sec:Bell_states}
\begin{figure}[h]
 \begin{center}
  \bgroup
   %Il bgroup è per avere un minimo di padding verticale nella tabella
   \def\arraystretch{1.3}
   \begin{tabular}{ l | c }
       \hline
       In & Out  \\ \hline
       $\ket{00}$ & $\left( \ket{00} + \ket{11} \right)/\sqrt{2} =: \ket{\beta_{00}}$  \\
       $\ket{01}$ & $\left( \ket{01} + \ket{10} \right)/\sqrt{2} =: \ket{\beta_{01}}$  \\
       $\ket{10}$ & $\left( \ket{00} - \ket{11} \right)/\sqrt{2} =: \ket{\beta_{10}}$  \\
       $\ket{11}$ & $\left( \ket{01} - \ket{10} \right)/\sqrt{2} =: \ket{\beta_{11}}$  \\
       \hline
   \end{tabular}
  \egroup
  \quad \quad
  \begin{quantikz}[column sep=1.8em, row sep=2.2em, thin lines]
   \lstick{$\ket{x}$} & \gate{H} & \ctrl{1} & \qw\rstick[wires=2]{$\ket{\beta_{xy}}$} \\
   \lstick{$\ket{y}$} & \qw & \targ{} & \rstick{} \qw
  \end{quantikz}
 \end{center}
 \caption{Circuito che costruisce gli stati di Bell a partire dagli elementi della base computazionale. Nella tabella sono elencati gli output del circuito, al variare dei valori in input.}\label{fig:Bell_states}
\end{figure}
A partire dagli elementi della base computazionale il circuito rappresentato in figura \ref{fig:Bell_states} produce i seguenti stati:
\begin{align}
 \ket{\beta_{00}} &= \frac{\ket{00} + \ket{11}}{\sqrt{2}}\\
 \ket{\beta_{01}} &= \frac{\ket{01} + \ket{10}}{\sqrt{2}}\\
 \ket{\beta_{10}} &= \frac{\ket{00} - \ket{11}}{\sqrt{2}}\\
 \ket{\beta_{11}} &= \frac{\ket{01} - \ket{10}}{\sqrt{2}}
\end{align}
in cui la notazione $\ket{\beta_{xy}}$, oltre a rappresentare gli stati di partenza su cui opera il circuito \ref{fig:Bell_states}, può essere usata per ottenere l'esplicita espressione del corrispondente stato di Bell tramire la seguente espressione
\begin{align}
 \ket{\beta_{xy}} = \frac{\ket{0y} + (-1)^x \ket{1\bar{y}}}{\sqrt{2}},
\end{align}
dove $\bar{y} := \neg y$ è la negazione di $y$.

Questi stati sono noti con il nome di \textit{stati di Bell} o \textit{stati EPR} in onore di Bell, Einstein, Podolsky e Rosen, i quali per primi hanno studiato le proprietà di questi speciali stati.
Essi, infatti, sono stati entangled, come è stato mostrato per $\ket{\beta_{00}}$ nella sezione \ref{sec:entanglement}, e formano una base per $\C{4}$.
Quest'ultima affermazione, in particolare, è evidente dal fatto che sono immagini tramite un circuito quantistico, ovvero un operatore unitario, degli elementi della base computazionale.

Questi stati, inoltre, sono legati dalle seguenti relazioni, che implicitamente giocheranno un ruolo molto importante nelle prossime sezioni:
\begin{align}\label{eqn:Bell_identities}
 Id \otimes Id \ket{\beta_{00}} = Id \otimes Id \ket{\beta_{00}} = \ket{\beta_{00}};&\
 X \otimes Id \ket{\beta_{00}} = Id \otimes X \ket{\beta_{00}} = \ket{\beta_{01}};\\
 Z \otimes Id \ket{\beta_{00}} = Id \otimes Z \ket{\beta_{00}} = \ket{\beta_{10}};&\
 iY \otimes Id \ket{\beta_{00}} = Id \otimes iY \ket{\beta_{00}} = \ket{\beta_{11}}
\end{align}
dove $Id, X, Y$ e $Z$ sono le matrici di Pauli, per cui
\begin{align}
 Id  = 
 \begin{bmatrix}
  1 & 0\\
  0 & 1
 \end{bmatrix};\ \
 X  = 
 \begin{bmatrix}
  0 & 1\\
  1 & 0
 \end{bmatrix};\ \
 Z = 
 \begin{bmatrix}
  1 & 0\\
  0 & -1
 \end{bmatrix};\ \
 iY = ZX = 
 \begin{bmatrix}
  0 & 1\\
  -1 & 0
 \end{bmatrix}.
\end{align}


\subsection{Teletrasporto quantistico}\label{sec:QTeleportation}
Si sfrutteranno, ora, le peculiarità degli stati di Bell per ottenere un risultato sorprendente e di grande utilità per la comprensione del funzionamento dei computer quantistici. 
L'obiettivo del seguente algoritmo, detto di \textit{teletrasporto quantistico}, è di trasmettere lo stato $\ket{\psi}$ di un qubit tra due computer quantistici, rispettivamente in mano ad Alice e Bob, scambiando solamente 2 bit di informazione classica.
Senza un precedente scambio di informazioni tale processo è chiaramente impossibile, in quanto, come evidenziato nelle sezioni precedenti, lo stato di un qubit è descritto da 3 valori reali, che richiederebbero una quantità infinita di bit classici per essere descritti. 

\begin{figure}
 \begin{center}
  \begin{quantikz}[column sep=2em, row sep=1.6em, thin lines, slice style=black]
   \lstick{$\ket{\psi}$} & \qw \slice{$\ket{\psi_0}$} & \ctrl{1} \slice{$\ket{\psi_1}$} & \gate{H} \slice{$\ket{\psi_2}$} & \meter{} \rstick{$\ M_1$\\\vspace{1.5mm}} & \cw \slice{$\ket{\psi_3}$} & \cw  & \cwbend{2} \slice{$\ket{\psi_4}$} & \\
   \lstick[wires=2]{$\ket{\beta_{00}}$} & \qw & \targ{} & \qw & \meter{} \rstick{$\ M_2$\\\vspace{1.5mm}} & \cw  & \cwbend{1} & \\
   & \qw & \qw & \qw & \qw & \qw & \gate{X^{M_2}} & \gate{Z^{M_1}} & \rstick{$\ket{\psi}$}\qw 
  \end{quantikz}
 \end{center}
 \caption{Circuito per l'algoritmo di teletrasporto quantistico. Lo stato $\ket{\psi}$ passa dal primo all'ultimo registro, tramite sola comunicazione classica.}\label{crq:Quantum_teleportation}
\end{figure}

Per sorvolare questo problema si crea uno stato entangled di Bell, per semplicità $\ket{\beta_{00}}$, e si dà il primo qubit di questa coppia ad Alice e il secondo a Bob.
Per quanto i due computer quantistici possano essere separati da una grande distanza spaziale, si descriverà il sistema con un unico stato dei 3-qubit, in cui, inizialmente, il primo (da trasmettere) sarà nello stato $\ket{\psi} = a \ket{0} + b \ket{1}$ e gli altri due in quello entangled $\ket{\beta_{00}}$.
Di questo stato i valori $a$ e $b$ sono arbitrari e, a priori, sconosciuti.
Si assume, inoltre, che i primi due qubit siano in possesso di Alice e solo il terzo di Bob.
Lo stato iniziale, quindi, sarà
\begin{equation}
 \ket{\psi_0} := \ket{\psi}\ket{\beta_{00}} = \frac{1}{\sqrt{2}}\left\{ a\ket{0} \left(\ket{00} + \ket{11} \right) + b\ket{1} \left(\ket{00} + \ket{11} \right) \right\},
\end{equation}
come indicato in figura \ref{crq:Quantum_teleportation}. Inizialmente Alice agisce sulla propria coppia di qubit, prima con un gate \textsc{cnot}, ad ottenere lo stato
\begin{equation}
 \ket{\psi_1} = \frac{1}{\sqrt{2}}\left\{a \ket{0} \left(\ket{00} + \ket{11} \right) + b \ket{1} \left(\ket{10} + \ket{01} \right) \right\}
\end{equation}
e, poi, con un gate Hadamard sul primo qubit, ad ottenere
\begin{equation}
 \ket{\psi_2} = \frac{1}{2} \left\{a \left(\ket{0} + \ket{1} \right) \left(\ket{00} + \ket{11} \right) + b \left(\ket{0} - \ket{1} \right) \left(\ket{10} + \ket{01} \right) \right\}.
\end{equation}
Questo stato, in particolare, può essere riscritto raggruppando i primi 2 qubit, in possesso di Alice, come segue
\begin{align}
 \ket{\psi_2} = \frac{1}{2}\{&\ket{00} \left(a\ket{0} + b\ket{1} \right) + \ket{01} \left(a\ket{1} + b\ket{0} \right) \nonumber \\
 + &\ket{10} \left(a\ket{0} - b\ket{1} \right) + \ket{11} \left(a\ket{1} - b\ket{0} \right) \}.
\end{align}
In questo stato, Alice, misurando la propria coppia di qubit, forzerà, a causa dell'entanglement della coppia di Bell, il bit di Bob ad assumere uno dei $4$ stati appena elencati. 
In particolare, in funzione della misurazione $M_1M_2$ di Alice, lo stato del qubit di Bob sarà $\ket{\psi_3(M_1M_2)}$, più esplicitamente:
\begin{align}\label{eqn:QT_post_measure1}
 M_1M_2 = 00 \implies \ket{\psi_3(00)} &= \left( a \ket{0} + b \ket{1} \right)\\
 M_1M_2 = 01 \implies \ket{\psi_3(01)} &= \left( a \ket{1} + b \ket{0} \right)\\
 M_1M_2 = 10 \implies \ket{\psi_3(10)} &= \left( a \ket{0} - b \ket{1} \right)\\
 M_1M_2 = 11 \implies \ket{\psi_3(11)} &= \left( a \ket{1} - b \ket{0} \right).\label{eqn:QT_post_measure4}
\end{align}
A questo punto, in seguito alla comunicazione della coppia di bit dalla misurazione di Alice dei propri qubit, Bob può provvedere ad operare sul qubit in suo possesso per ottenere $\ket{\psi}$.
In particolare dovrà agire con le seguenti operazioni:
\begin{align}
 M_1M_2 = 00 \implies Id\ket{\psi_3(00)} =& \left( a \ket{0} + b \ket{1} \right) = \ket{\psi}\\
 M_1M_2 = 01 \implies X\ket{\psi_3(01)} =& \left( a \ket{0} + b \ket{1} \right) = \ket{\psi}\\
 M_1M_2 = 10 \implies Z\ket{\psi_3(10)} =& \left( a \ket{0} + b \ket{1} \right) = \ket{\psi}\\
 M_1M_2 = 11 \implies ZX\ket{\psi_3(11)} =& \left( a \ket{0} + b \ket{1} \right) = \ket{\psi}.
\end{align}
In particolare, nell'ultima riga si opera con la trasformazione $ZX$, che corrisponde ad applicare prima il gate $X$ e poi il gate $Z$, come rappresentato nel circuito \ref{crq:Quantum_teleportation}, in cui le operazioni si susseguono da sinistra a destra.
Inoltre risulta chiaro dal prospetto precedente che le operazioni corrette per ottenere lo stato $\ket{\psi}$, data la misurazione $M_1M_2$, sono $Z^{M_1}X^{M_2}$, ovvero quanto rappresentato nel circuito, in cui lo stato finale, correttamente, è proprio $\ket{\psi}$.

Questo interessante fenomeno merita due ulteriori commenti prima di essere abbandonato.
Innanzitutto potrebbe parre che, trasportando lo stato del qubit $\ket{\psi}$ da Alice a Bob, se ne stia creando una copia, in violazione del teorema \ref{thm:N-C} (no cloning theorem).
Ciò non è vero, in quanto, in seguito alla misurazione della coppia di qubit di Alice, il primo registro, in cui era immagazzinato lo stato $\ket{\psi}$, si ritrova in uno dei due stati della base computazionale: $\ket{0}$ o $\ket{1}$.
Si ottiene, quindi, una sola copia dello stato $\ket{\psi}$, senza alcuna contraddizione del suddetto teorema.

Un'ulteriore commento, di natura più fisica, riguarda la parvenza che questo circuito permetta, in contraddizione con la teoria della relatività generale, la trasmissione di informazione a velocità superluminale.
Questa idea potrebbe essere scatenata dal fatto che la misurazione dei qubit di Alice porta all'immediato ``collasso'' del qubit di Bob a uno degli stati in \eqref{eqn:QT_post_measure1}-\eqref{eqn:QT_post_measure4}. 
Anche questa affermazione è solo apparentemente vera: per ottenere lo stato del primo qubit Bob deve attendere l'informazione classica della misurazione effettuata da Alice, la quale non può essere trasmessa a velocità superluminali.
In realtà, con strumenti matematici più potenti rispetto a quelli introdotti in questa breve trattazione, si può mostrare che, senza lo scambio dei bit classici, il solo fenomeno di entaglement non trasferisce alcuna informazione -- si veda \cite{Book:QCQI} alla sezione 2.4.3.

\subsection{Superdense coding}
\begin{figure}[h]
 \begin{center}
  \begin{quantikz}[column sep=1.5em, row sep=1.6em, thin lines,slice style=black]
   \lstick{$\ket{0}$} & \gate{H}\gategroup[2,steps=2,style={dashed, inner xsep=2pt}, background, label style={label position=below,anchor=north,yshift=-0.2cm}]{Preparazione} & \ctrl{1} & \qw \slice{$\ket{\beta_{00}}$} & \qw & \gate{X^{b_2}}\gategroup[1,steps=2,style={dashed, inner xsep=2pt}, background, label style={label position=above,anchor=south,yshift=-0.2cm}]{Alice} & \gate{Z^{b_1}} & \qw\slice{$\ket{\beta_{b_1b_2}}$} & \qw & \ctrl{1}\gategroup[2,steps=5,style={dashed, inner xsep=2pt}, background, label style={label position=below,anchor=north,yshift=-0.2cm}]{Bob} & \gate{H}\slice{$\ket{b_1b_2}$} & \meter{} & \cw \rstick{$b_1$} & \\
   \lstick{$\ket{0}$} & \qw & \targ{} & \qw & \qw &\qw\gategroup[1,steps=4,style={dashed, inner xsep=2pt}, background, label style={label position=below,anchor=north,yshift=-0.2cm}]{Bob} & \qw & \qw & \qw & \targ{} & \qw & \meter{}  & \cw \rstick{$b_2$} &
  \end{quantikz}
 \end{center}
 \caption{Circuito per l'algoritmo di superdense coding. Permette la trasmissione di 2 bit classici tramite lo scambio di 1 solo qubit.}\label{crq:Superdense_coding}
\end{figure}

L'algoritmo precedente ha una controparte, che sfrutta sempre gli stati entangled di Bell, per ottenere l'effetto opposto: inviare due bit classici $b_1$ e $b_2$, tramite lo scambio di un singolo qubit da Alice a Bob.
Tale algoritmo è chiamato \textit{superdense coding} ed è rappresentato nel circuito in figura \ref{crq:Superdense_coding}.
Per comprenderne appieno il funzionamento si commentano i vari stadi del circuito.
Nel riquadro ``Preparazione'', a partire da $\ket{00}$, viene preparato lo stato $\ket{\beta_{00}}$, come descritto in sezione \ref{sec:Bell_states}.
Il primo qubit di questo stato viene dato ad Alice, mentre il secondo a Bob.
Come descritto dalle identità in \eqref{eqn:Bell_identities} Alice trasforma il qubit $\ket{\beta_{00}}$ in $\ket{\beta_{b_1b_2}}$, dove $b_1$ e $b_2$ sono i bit classici che vuole trasmettere a Bob.
A questo punto Alice invia anche il proprio qubit a Bob, il quale è, ora, in possesso di entrambi i componenti dello stato entangled.
Per reversibilità degli operatori quantistici Bob, eseguendo al contrario il circuito quantistico in figura \ref{fig:Bell_states} ottiene, a partire da $\ket{\beta_{b_1b_2}}$, lo stato della base computazionale $\ket{b_1b_2}$.
A Bob non resta che misurare lo stato in proprio possesso nella base computazionale per ottenere, con certezza, la coppia di bit $b_1b_2$ che Alice doveva comunicargli, come ci si era prefissati.

Gli ultimi due algoritmi mostrano come, fondamentalmente, la capacità di memoria di un sistema di $n$-qubit sia $2^n$ bit classici, dando alla computazione quantistica, su questo fronte, un vantaggio esponenziale rispetto a quella classica.

\subsection{Parallelismo quantistico}\label{sec:Quantum_par}
Una fondamentale caratteristica dei computer quantistici è la possibilità di valutare svariati valori di una funzione binaria contemporaneamente. Questa proprietà è chiamata \textit{parallelismo quantistico} ed è alla base di tutta la sezione \ref{sec:QFT}, in cui la trasformata di Fourier quantistica ne farà uso.

Per comprendere appieno le potenzialità, ma soprattutto le limitazioni, di questo peculiare comportamento quantistico è necessario osservarne il funzionamento nel dettaglio.
Per chiarezza sarà trattato per primo il caso di una funzione $f:\left\{ 0,1 \right\} \to \left\{ 0,1 \right\}$ arbitraria.
Questa funzione non è necessariamente unitaria (o invertibile), quindi non è immediatamente implementabile quantisticamente.
Si può, però, decidere di tenere in memoria sia il valore in entrata alla funzione, sia l'immagine della funzione.
Si crea, quindi, il gate $U_f$ che manda il generico stato $\ket{x,y}$ in $\ket{x,y \oplus f(x)}$, dove $\oplus$ indica la somma modulo 2. Per funzioni binarie come $f$, inoltre, tale trasformazione è unitaria.

In particolare, se $y=0$, lo stato $\ket{x,0}$ verrà mandato da $U_f$ in $\ket{x, f(x)}$, dando esplicitamente il valore della funzione. La parte interessante del procedimento per ottenere il parallelismo quantistico, però, si incontra analizzando $x$.
Infatti, ponendo $\ket{x} = \left(\ket{0} + \ket{1} \right)/\sqrt{2}$ -- facilmente preparabile applicando un gate $H$ a un qbit nello stato $\ket{0}$ -- $U_f$ agirà nel seguente modo:
\begin{equation}
 \ket{x,0} = \frac{\ket{0} + \ket{1}}{\sqrt{2}}\ket{0} \stackrel{U_f}{\longmapsto} \frac{\ket{0,f(0)} + \ket{1,f(1)}}{\sqrt{2}}.
\end{equation}
In questo stato sono presenti entrambe le immagini della funzione $f$ e queste sono state computate simultaneamente, da un singolo circuito.

Analogamente, data una funzione $f:\left\{ 0,1 \right\}^n \to \left\{ 0,1 \right\}^m$, si definisce un gate $U_f$, che opera come
\begin{equation}
 \ket{x,y} \stackrel{U_f}{\longmapsto} \ket{x,y + f(x) \mod 2^m}.
\end{equation}
Scrivendone la rappresentazione come matrice risulta chiaro che è anch'esso unitario. 
Si può persino mostrare che, se esiste un algoritmo classico per computare $f:\left\{ 0,1 \right\}^n \to \left\{ 0,1 \right\}^m$, allora l'algoritmo quantistico per computare $U_f$ è di paragonabile efficienza.
Per approfondire si rimanda a \cite{Book:QCQI}, sezione 3.2.5.\\
In particolare, a partire dallo stato 
\begin{equation}
 \ket{x}\ket{y} := \left(\frac{1}{2^{n/2}}\sum_{j=0}^{2^n-1} \ket{j}\right)\ket{0} = 
 \frac{1}{2^{n/2}}\sum_{j=0}^{2^n-1} \ket{j}\ket{0},
\end{equation}
il gate $U_f$ produrrà lo stato
\begin{equation}
 \frac{1}{2^{n/2}}\sum_{j=0}^{2^n-1} \ket{j}\ket{f(j)},
\end{equation}
il quale, come nel caso precedente, contiene tutte le immagini di $f$.
In particolare lo stato $\ket{x}$ iniziale può essere facilmente preparato applicando un gate Hadamard $H^{\otimes n}$ allo stato $\ket{0}{\otimes n}$.

A seguito di quest'analisi risulta chiaro che il circuito quantistico alla base del parallelismo quantistico corrisponda con quanto rappresentato in figura \ref{crq:Quantum_parallelism}.
Inoltre, per lo scopo attuale, non è importante conoscere la realizzazione del gate $U_f$, ma solo il fatto che possa essere realizzato.
Per l'applicabilità del processo di parallelismo può essere semplicemente considerato come una scatola nera.

\begin{figure}[h]
 \begin{center}
  \begin{quantikz}[column sep=2em, row sep=1.6em, thin lines]
   \lstick{$\ket{0}$} & \gate{H^{\otimes n}}\qwbundle{n} & \gate[wires=2][2cm]{U_f}\gateinput{$x$}\gateoutput{$x$} & \qw\rstick[wires=2]{$\displaystyle{\frac{1}{2^{n/2}}\sum_{x=0}^{2^n-1}\ket{x}\ket{f(x)}}$}\\
   \lstick{$\ket{0}$} & \qw\qwbundle{m} & \gateinput{$y$}\gateoutput{$y \oplus f(x)$} & \qw
  \end{quantikz}
 \end{center}
 \caption{Circuito per l'algoritmo di parallelismo quantistico. Crea uno stato finale contenente la valutazione della funzione $f$ su tutti gli elementi del proprio dominio.}
 \label{crq:Quantum_parallelism}
\end{figure}

Ora che è chiaro il funzionamento dell'algoritmo si possono analizzare più nel dettaglio le sue proprietà.
Innanzitutto, indipendentemente dalla grandezza del dominio della funzione $f$, supposto di aver abbastanza memoria per poter rappresentare tutti gli elementi del dominio, basta un singolo circuito quantistico per operare con il parallelismo. 
In ambito classico, per valutare la funzione su tutti gli elementi del dominio in contemporanea sarebbe necessario un circuito per ciascun elemento, con una richiesta di risorse fisiche esponenzialmente più grande.

Nonostante questo grandissimo vantaggio, purtroppo, lo stato finale di questo circuito non è immediatamente accessibile: per osservarlo sarebbe necessario misurarlo, facendolo collassare su un solo elemento della sovrapposizione.
Ciò restituirebbe come risultato l'immagine di un singolo elemento del dominio, scelto casualmente, cancellando il resto della computazione.
Questo fatto comporta la necessità di sfruttare in modo meno diretto la grande quantità di informazioni contenuta nella sovrapposizione.
È necessario, infatti, in base al problema da affrontare, operare ulteriormente sullo stato preparato dal parallelismo, per ottenere, con grande probabilità, un risultato corretto.
Per mostrare come ciò possa essere messo in atto, nella prossima sezione, verrà analizzato un problema la cui risoluzione quantistica si basa su un ingegnoso uso del parallelismo.

\subsection{Algoritmo di Deutsch-Josza}\label{sec:Deutsch-Josza}
L'algoritmo di \textit{Deutsch-Josza} si pone un problema di poca utilità pratica, ma grande utilità didattica.
Servirà per comprendere come il parallelismo quantistico possa essere utilizzato per risolvere problemi in cui è necessario valutare una funzione più volte.
Innanzitutto è necessario spiegare il problema da risolvere.
Sia $f:\left\{0,1\right\}^n \to \left\{0,1\right\}$ una funzione binaria.
Tale funzione deve, inoltre, essere \textit{costante}, ovvero assumere lo stesso valore per ogni elemento del dominio, oppure \textit{bilanciata}, ovvero asssumere il valore $1$ esattamente la metà delle volte e $0$ l'altra metà.
L'algoritmo di \textit{Deutsch-Josza}, con una sola esecuzione della scatola nera $U_f$ vuole individuare con certezza se la corrispondente funzione $f$ è bilanciata o costante.
Per ottenere tale risultato si opera, come descritto nel circuito della figura \ref{crq:Deutsch-Josza}, applicando un circuito di \textit{parallelismo quantistico} e manipolando lo stato ottenuto in modo tale da far interferire tra loro gli elementi della sovrapposizione, fino a poter misurare, con certezza, una risposta al problema.

\begin{figure}[h]
 \begin{center}
  \begin{quantikz}[column sep=2em, row sep=1.6em, thin lines, slice style=black]
   \lstick{$\ket{0}$} & \qw\qwbundle{n}\slice{$\ket{\psi_0}$} & \gate{H^{\otimes n}}\slice{$\ket{\psi_1}$} & \gate[wires=2][2cm]{U_f}\gateinput{$x$}\gateoutput{$x$}\slice{$\ket{\psi_2}$} & \gate{H^{\otimes n}}\slice{$\ket{\psi_3}$} & \meter{}\\
   \lstick{$\ket{1}$} & \qw & \gate{H} & \gateinput{$y$}\gateoutput{$y \oplus f(x)$} &  \qw & \qw
  \end{quantikz}
 \end{center}
 \caption{Circuito per l'algoritmo di Deutsch-Josza. Riesce a distinguere tra funzioni binarie costanti ed equilibrate.}
 \label{crq:Deutsch-Josza}
\end{figure}
Per comprendere la breve spiegazione precedente è necessario descrivere passo passo il comportamento del circuito. 
Lo stato iniziale della computazione è
\begin{equation}
 \ket{\psi_0} = \ket{0}^{\otimes n} \ket{1},
\end{equation}
su cui operano i gate Hadamard per creare la sovrapposizione
\begin{equation}
 \ket{\psi_1} = \sum_{x \in \left\{0,1\right\}^n} \frac{\ket{x}}{2^{n/2}} \left[\frac{\ket{0} - \ket{1}}{\sqrt{2}} \right].
\end{equation}
Questo stato è interessante in quanto i primi $n$ qubit, come nell'algoritmo di parallelismo quantistico, sono nella una sovrapposizione uniforme di tutti gli elementi del dominio.
L'ultimo bit, invece, è in una combinazione delle immagini possibili, non solo nello stato $\ket{0}$, come ci si aspetterebbe per operare con il parallelismo.
Questo fa intuire che si voglia sfruttare una particolare sovrapposizione per ottenere una risposta al problema.

Fissato $x \in \left\{0,1\right\}^n$, posto $y := \left(\ket{0} - \ket{1}\right)/\sqrt{2}$, l'azione di $U_f$ su $\ket{x,y}$ diventa
\begin{equation}
 \ket{x,y} \stackrel{U_f}{\longmapsto} \ket{x} \left[ \frac{\ket{0 \oplus f(x)} - \ket{1 \oplus f(x)}}{\sqrt{2}} \right] = (-1)^{f(x)}\ket{x} \left[ \frac{\ket{0} - \ket{1}}{\sqrt{2}} \right].
\end{equation}
Segue che, dopo l'applicazione del gate $U_f$, lo stato del circuito diventa
\begin{equation}
 \ket{\psi_2} = \sum_{x \in \left\{0,1\right\}^n} \frac{(-1)^{f(x)}\ket{x}}{2^{n/2}} \left[\frac{\ket{0} - \ket{1}}{\sqrt{2}} \right].
\end{equation}
In questo stato l'informazione dell'immagine di $f$ non è accessibile solo dall'ultimo qubit, ma anche dai primi $n$. 
Si può, quindi, operare per fare interferire i termini della combinazione lineare dei primi $n$ qubit per ottenere una risposta al problema. 
Per raggiungere questo obiettivo, nel circuito rappresentato, si opera con un gate di Hadamard, arrivando allo stato $\ket{\psi_3}$.
Al fine di  calcolarne la forma esplicita si ricorda l'espressione per $H^{\otimes n}$ individuata in \eqref{eqn:H^n}, per cui, dato $z \in \left\{0,1\right\}^n$,
\begin{equation}
 H^{\otimes n}\ket{z} = \frac{1}{2^{n/2}}\sum_{x \in \left\{0,1\right\}^n} \left(-1 \right)^{x\cdot z}\ket{x},
\end{equation}
dove $x \cdot z$ rappresenta il prodotto bit per bit, modulo 2, tra numeri binari.
Questa rappresentazione si presta molto bene alla scrittura esplicita di $\ket{\psi_3}$, il quale risulta essere
\begin{equation}
 \ket{\psi_3} = \sum_{z \in \left\{0,1\right\}^n} \sum_{x \in \left\{0,1\right\}^n} \frac{(-1)^{x\cdot z + f(x)}\ket{z}}{2^n} \left[\frac{\ket{0} - \ket{1}}{\sqrt{2}} \right].
\end{equation}
In cui, in particolare, il coefficiente dello stato $\ket{0}^{\otimes n}$ del primo registro è $\left(\sum_x (-1)^{f(x)} \right)/2^n$. 
In base al fatto che $f$ sia costante o bilanciata questa espressione assumerà due possibili valori. 
Se $f$ è costante, allora il termine $\ket{0}^{\otimes n}$ avrà coefficiente $+1$ o $-1$, in base al valore di $f$. 
Segue che ogni altro termine nella combinazione lineare dello stato $\ket{\psi_3}$ sarà 0.
Se, invece, $f$ è bilanciata allora lo stato $\ket{0}^{\otimes n}$ avrà coefficiente 0, in quanto i termini positivi e negativi sono in egual numero.
In particolare, eseguendo una misura nella base computazionale dei primi $n$ qubit, se $f$ è costante si otterrà con certezza 0, altrimenti, con certezza, si otterrà un risultato diverso da 0, permettendo di distinguere tra i due casi possibili.
Si è riusciti, quindi, a far interferire i termini della sovrapposizione in modo tale da ottenere facilmente una risposta a un problema richiedente la valutazione di una funzione su tutti gli elementi del dominio.
Per riassumere, a meno dei commenti, segue una descrizione dell'algoritmo di Deutsch-Josza.

\begin{algo}[Deutsch-Josza]\*
\begin{description}
  \item [Input:] (1) una scatola nera per eseguire il gate $U_f$, che opera su $x \in \left\{0,1\right\}^n$ e $y \in \left\{0,1\right\}$ come $\ket{x}\ket{y} \mapsto \ket{x}\ket{y\oplus f(x)}$. Nel gate precedente la funzione è $f: \left\{0,1\right\}^n \to \left\{0,1\right\}$ ed è bilanciata o costante. (2) $n$ qubit inizializzati nello stato $\ket{0}$ ed un qubit nello stato $\ket{1}$.
  \item [Output:] 0 se la funzione $f$ è costante, un valore diverso da 0 se è bilanciata.
  \item [Runtime:] una singola chiamata alla scatola nera per l'esecuzione del gate $U_f$. Ritorna sempre un output corretto.
  \newpage
  \item [Procedura:] \ 
    \begin{enumerate}
     \item ${\displaystyle\ket{0}^{\otimes n}\ket{1}}$
     \hfill {\small stato iniziale}
     \item ${\displaystyle\stackrel{H^{\otimes n+1}}{\longmapsto} \frac{1}{2^{n/2}} \sum_{x=0}^{2^n-1} \ket{x}\left[\frac{\ket{0} - \ket{1}}{\sqrt{2}} \right]}$
     \hfill {\small sovrapposizione iniziale}
     \item ${\displaystyle\stackrel{U_f}{\longmapsto} \sum_{x=0}^{2^n-1} \frac{(-1)^{f(x)}\ket{x}}{2^{n/2}} \left[\frac{\ket{0} - \ket{1}}{\sqrt{2}} \right]}$
     \hfill {\small risultato scatola nera}
     \item ${\displaystyle\stackrel{H^{\otimes n}}{\longmapsto} \sum_{z=0}^{2^n-1} \sum_{x=0}^{2^n-1} \frac{(-1)^{x\cdot z + f(x)}\ket{z}}{2^n} \left[\frac{\ket{0} - \ket{1}}{\sqrt{2}} \right]}$
     \hfill {\small sovrapposizione finale}
     \item ${\displaystyle\longmapsto z}$
     \hfill {\small misura primo registro}
    \end{enumerate}
 \end{description}
\end{algo}

\section{Trasformata di Fourier quantistica} \label{sec:QFT}
Classicamente si definisce la \textit{trasformata di Fourier discreta} l'applicazione che associa al vettore $\left(x_0, \dotsc, x_{N-1}\right)$ di lunghezza $N$ il vettore $\left(y_0, \dotsc, y_{N-1}\right)$, le cui componenti sono determinate dalla seguente espressione:
\begin{equation}\label{eqn:y_k}
 y_k := \frac{1}{\sqrt{N}} \sum_{j=0}^{N-1} x_j e^{2 \pi i j k/N}.
\end{equation}

Analogamente la \textit{trasformata di Fourier quantistica} o \textsc{qft} (Quantum Fourier Transform) è l'unico operatore lineare così definito sulla base ortonormale $\left(\ket{0}, \dotsc, \ket{N-1}\right)$:
\begin{equation}\label{defn:QFT}
 \ket{j} \mapsto \frac{1}{\sqrt{N}} \sum_{k=0}^{N-1} e^{2 \pi i j k/N} \ket{k}.
\end{equation}
Da cui si ottiene l'azione della \textsc{qft} sul generico stato:
\begin{equation}\label{defn:QFT_generica}
 \sum_{j=0}^{N-1}x_j\ket{j} \mapsto \sum_{k=0}^{N-1} y_k \ket{k},
\end{equation}
dove le $y_k$ sono le immagini tramite la \textit{trasformata di Fourier discreta} del vettore $\left(x_0, \dotsc, x_{N-1}\right)$, definite in \eqref{eqn:y_k}.

Non è ovvio dalla definizione, ma tale operatore è unitario:

\begin{lem}
 L'operatore \textit{trasformata di Fourier quantistica} è unitario.
\end{lem}
\begin{proof}
 Dalla definizione della \textit{trasformata di Fourier quantistica} risulta chiaro che, nella base $\left(\ket{0}, \dotsc, \ket{N-1}\right)$, ha rappresentazione matriciale data da:
 
 \begin{equation}\label{defn:QFT_Matrix}
  F := \frac{1}{\sqrt{N}} \left[e^{2 \pi i j k / N} \right]_{j,k}
 \end{equation}
 che, per commutatività del prodotto, è chiaramente una matrice simmetrica. 
 In quanto, inoltre, $(e^{2 \pi i j k / N})^* = e^{-2 \pi i j k / N}$, si trova che la matrice associata all'operatore aggiunto di $F$ è:
 
 \begin{equation}\label{defn:QFTa_Matrix}
  F^\dag = \frac{1}{\sqrt{N}} \left[e^{-2 \pi i j k / N} \right]_{j,k}.
 \end{equation}
 Prendendo il prodotto di \eqref{defn:QFT_Matrix} e \eqref{defn:QFTa_Matrix} si ottiene:
 \begin{equation}\label{eqn:U_QFT}
 \left[F \cdot F^{\dagger} \right]_{l,m} = \frac{1}{N} \sum_{j=0}^{N-1} e^{\frac{2 \pi i}{N} j \cdot \left(l - m \right)}.
 \end{equation}
 Se $l = m$, l'ultimo termine di \eqref{eqn:U_QFT} diventa 
 \begin{equation}\label{eqn:U_QFT_ausiliario1}
  {\frac{1}{N} \sum_{j=0}^{N-1} 1} = \frac{1}{N} \cdot N = 1,
 \end{equation}
 altrimenti ${l - m \neq 0}$, per cui, al variare di ${j \in \left\{0, \dotsc, N-1 \right\}}$, i termini $e^{\frac{2 \pi i}{N} j \cdot \left(l - m \right)}$ scorrono, magari più di una volta, tutte le radici $R$-esime dell'unità, ove $R := N /$ MCD$\left(N,l-m\right)$. Inoltre, per ogni $R \in \mathbb{N}$, vale la scomposizione:
 \begin{equation}
  x^{R} - 1 = (x - 1) (x^{R - 1} + x^{R - 2} + \dotsb + 1).
 \end{equation}
 Se $z$ è una radice $R$-esima primitiva dell'unità, allora
 \begin{equation}\label{eqn:Sum_Rth_1_rad}
  z^R - 1 = 0 = (z - 1) (z^{R - 1} + z^{R - 2} + \dotsb + 1)
 \end{equation}
 da cui, per la legge di annullamento del prodotto, segue che, se $l \neq m$, 
 \begin{equation}\label{eqn:U_QFT_ausiliario2}
  \left[F \cdot F^\dag \right]_{l,m} = \frac{1}{N} \sum_{j=0}^{N-1} e^{\frac{2 \pi i}{N} j \cdot \left(l - m \right)} = 0.
 \end{equation}
 Da \eqref{eqn:U_QFT_ausiliario1} e \eqref{eqn:U_QFT_ausiliario2} si può concludere \eqref{eqn:U_QFT} con
 \begin{equation}\label{eqn:U_QFT2}
 \left[F \cdot F^\dag \right]_{l,m} = \frac{1}{N} \sum_{j=0}^{N-1} e^{\frac{2 \pi i}{N} j \cdot \left(l - m \right)} = \delta_{l,m}.
 \end{equation}
 
 In particolare vale \eqref{eqn:U_QFT2} $\iff F \cdot F^\dag = Id \iff$ $F$ è un operatore unitario.
\end{proof}

Questo lemma ha come conseguenza il fatto che la trasformata di Fourier quantistica sia un gate quantistico ammissibile su $n$-qubit.
In particolare, adesso cercheremo di individuare esplicitamente un circuito quantistico per il calcolo della \textsc{qft}

\subsubsection{Circuito quantistico}
Si consideri $N = 2^n$, con $n \in \mathbb{N}$, e un sistema di $n$-qubit sul cui spazio degli stati si prende la base computazionale $\ket{0 \dots 0}, \dotsc, \ket{1 \dots 1}$. 
Usando la rappresentazione binaria per gli elementi della base $j = j_1 2^{n-1} + j_2 2^{n-2} + \dots + j_n 2^0 =: j_1j_2 \dots j_n$ e per le frazioni binarie $0.j_lj_{l+1} \dots j_m := j_l/2 + j_{l+1}/2^2 + \dots + j_m/2^{m-l+1}$, si ottiene una più agevole forma della \textsc{qft}:

\begin{lem}[rappresentazione prodotto]
 La trasformata di Fourier quantistica ha la seguente rappresentazione euqivalente:
 \begin{equation}\label{lem:QFT_PR}
 \ket{j_1j_2 \dots j_n} \mapsto \frac{\left( \ket{0} + e^{2 \pi i 0.j_n} \ket{1} \right) \left( \ket{0} + e^{2 \pi i 0.j_{n-1}j_n} \ket{1} \right) \dotsm \left( \ket{0} + e^{2 \pi i 0.j_1j_2 \dots j_n} \ket{1} \right)}{2^{n/2}}
\end{equation}
\end{lem}
\begin{proof}
 Per definizione \eqref{defn:QFT} la trasformata di Fourier quantistica agisce sugli elementi della base computazionale come:
 \begin{align*}
  \ket{j} &\mapsto \frac{1}{2^{n/2}} \sum_{k=0}^{2^n-1} e^{2 \pi i j k/2^n} \ket{k}\\
  & = \frac{1}{2^{n/2}} \sum_{k_1=0}^1 \sum_{k_2=0}^1 \dots \sum_{k_n=0}^1 e^{2 \pi i j \left( \sum_{l=1}^n k_l 2^{-l} \right)} \ket{k_1k_2 \dots k_n}\\
  & = \frac{1}{2^{n/2}} \sum_{k_1=0}^1 \sum_{k_2=0}^1 \dots \sum_{k_n=0}^1 \bigotimes_{l=1}^n e^{2 \pi i j k_l 2^{-l} } \ket{k_l}\\
  & = \frac{1}{2^{n/2}} \bigotimes_{l=1}^n \left[ \sum_{k_l=0}^1 e^{2 \pi i j k_l 2^{-l} } \ket{k_l} \right]\\
  & = \frac{1}{2^{n/2}} \bigotimes_{l=1}^n \left[ \ket{0} + e^{2 \pi i j 2^{-l} } \ket{1} \right]\\
  & = \frac{\left( \ket{0} + e^{2 \pi i 0.j_n} \ket{1} \right) \left( \ket{0} + e^{2 \pi i 0.j_{n-1}j_n} \ket{1} \right) \cdots \left( \ket{0} + e^{2 \pi i 0.j_1j_2 \dots j_n} \ket{1} \right)}{2^{n/2}}
 \end{align*}
\end{proof}

Per arrivare alla costruzione di un circuito quantistico per la \textsc{qft} si definisce il gate $R_k$:
\begin{equation} \label{defn:gate_R_k}
 R_k := 
 \begin{bmatrix}
      1 &   0   \\
      0 &   e^{2 \pi i/2^k}   \\
 \end{bmatrix},
\end{equation}
il quale, insieme con il gate di Hadamard è alla base del circuito rappresentato in figura \ref{crq:QFT}, che è il cuore dell'implementazione della \textit{trasformata di Fourier quantistica}.
\begin{figure}[h]
 \centering
\begin{adjustwidth}{-2cm}{}
 \begin{quantikz}[column sep=0.36em, row sep=0.48em, thin lines]
     \lstick{$\ket{j_1}$} & \gate{H} & \gate{R_2} & \qw & \ \ldots\ \qw & \gate{R_{n-1}} & \gate{R_n} & \qw & \qw & \ \ldots\ \qw & \qw & \qw & \qw & \ \ldots\ \qw & \qw & \qw & \qw & \qw\rstick{$\ket{0} + e^{2 \pi i 0.j_1 \dots j_n} \ket{1}$}\\
     \lstick{$\ket{j_2}$} & \qw & \ctrl{-1} & \qw & \ \ldots\ \qw & \qw & \qw & \gate{H} & \qw & \ \ldots\ \qw & \gate{R_{n-2}} & \gate{R_{n-1}} & \qw & \ \ldots\ \qw & \qw & \qw & \qw & \qw\rstick{$\ket{0} + e^{2 \pi i 0.j_2 \dots j_n} \ket{1}$}\\
     \lstick{$\vdots \ \ $} & & & & \vdots & & & & & \vdots & & & & \vdots & & & & \rstick{$\quad\ \ \vdots$}\\
     \lstick{$\ket{j_{n-1}}$} & \qw & \qw & \qw & \ \ldots\ \qw & \ctrl{-3} & \qw & \qw & \qw & \ \ldots\ \qw & \ctrl{-2} & \qw & \qw & \ \ldots\ \qw & \gate{H} & \gate{R_2} & \qw & \qw\rstick{$\ket{0} + e^{2 \pi i 0.j_{n-1}j_n} \ket{1}$}\\
     \lstick{$\ket{j_n}$} & \qw & \qw & \qw & \ \ldots\ \qw & \qw & \ctrl{-4} & \qw & \qw & \ \ldots\ \qw & \qw & \ctrl{-3} & \qw & \ \ldots\ \qw & \qw & \ctrl{-1} & \gate{H} & \qw\rstick{$\ket{0} + e^{2 \pi i 0.j_n} \ket{1}$}
 \end{quantikz}
\end{adjustwidth} 
 \caption{Circuito che implementa la trasformata di Fourier}
 \label{crq:QFT}
\end{figure}

Per comprendere come opera il circuito appena rappresentato lo seguiremo in dettaglio, a partire dallo stato iniziale $\ket{j_1j_2 \dots j_n}$. Applicando il gate di Hadamard al primo qubit si ottiene lo stato
\begin{equation}
 \frac{1}{2^{1/2}} \left( \ket{0} + e^{2 \pi i 0.j_1} \ket{1} \right) \ket{j_2 \dots j_n},
\end{equation}
in quanto $e^{2 \pi i 0.j_1} = -1$ se $j_1 = 1$, mentre vale $+1$ se $j_1 = 0$.
In seguito all'azione del gate $R_2$-controllato si arriva allo stato
\begin{equation}
 \frac{1}{2^{1/2}} \left( \ket{0} + e^{2 \pi i 0.j_1j_2} \ket{1} \right) \ket{j_2 \dots j_n}
\end{equation}
e, continuando ad applicare i gate $R_k$, con $k \in \left\{3, \dots, n \right\}$, al primo qubit si giunge a
\begin{equation}
 \frac{1}{2^{1/2}} \left( \ket{0} + e^{2 \pi i 0.j_1j_2 \dots j_n} \ket{1} \right) \ket{j_2 \dots j_n}.
\end{equation}
Analogamente si opera con un gate Hadamard sul secondo qubit, ottenendo
\begin{equation}
 \frac{1}{2^{2/2}} \left( \ket{0} + e^{2 \pi i 0.j_1j_2 \dots j_n} \ket{1} \right) \left( \ket{0} + e^{2 \pi i 0.j_2} \ket{1} \right) \ket{j_3 \dots j_n}
\end{equation}
che, in seguito all'azione dei gate $R_k$-controllati diventa
\begin{equation}
 \frac{1}{2^{2/2}} \left( \ket{0} + e^{2 \pi i 0.j_1j_2 \dots j_n} \ket{1} \right) \left( \ket{0} + e^{2 \pi i 0.j_2 \dots j_n} \ket{1} \right) \ket{j_3 \dots j_n}.
\end{equation}
Operando in modo analogo per i rimanenti qubit si ottiene lo stato finale
\begin{equation}
 \frac{\left( \ket{0} + e^{2 \pi i 0.j_1j_2 \dots j_n} \ket{1} \right) \left( \ket{0} + e^{2 \pi i 0.j_2 \dots j_n} \ket{1} \right) \cdots \left( \ket{0} + e^{2 \pi i 0.j_n} \ket{1} \right)}{2^{n/2}}.
\end{equation}

Questa non è ancora un'implementazione della \textsc{qft}. Operando con opportuni scambi di qubit, di cui il circuito \eqref{crq:Qswap} mostra una possibile realizzazione, si può ottenere un circuito il cui stato finale è
\begin{equation}
 \frac{\left( \ket{0} + e^{2 \pi i 0.j_n} \ket{1} \right) \left( \ket{0} + e^{2 \pi i 0.j_{n-1}j_n} \ket{1} \right) \cdots \left( \ket{0} + e^{2 \pi i 0.j_1j_2 \dots j_n} \ket{1} \right)}{2^{n/2}}.
\end{equation}
che, come definito in \eqref{lem:QFT_PR}, è l'immagine cercata.
Si ha un circuito che implementa la \textit{trasformata di Fourier quantistica}.

È interessante notare che, per $k$ molto grandi, i gate $R_j$, con $j$ piccolo, operano con un piccolo cambiamento di fase.
Fisicamente risultano molto difficili da realizzare in modo accurato, per cui sono stati studiati circuiti quantistici che implementino la trasformazione in modo approssimato, evitando i gate incriminati.

\subsubsection{Costo computazionale}
Il circuito che implementa la \textit{trasformata di Fourier quantistica}, basato su quello rappresentato in \eqref{crq:QFT}, richiede il seguente numero di gate per essere implementato:
\begin{enumerate}
 \item In \eqref{crq:QFT} sono utilizzati $1$ gate Hadamard ed $n-1$ gate $R_k$ per il primo qubit, $n-1$ gate per il secondo qubit e così via fino all'$n$-esimo qubit con $1$ gate, per un totale di $n + (n-1) + \dots + 1 = \frac{n(n+1)}{2}$ gate.
 \item Gli scambi finali sono al più $n/2$, ciascuno dei quali è implementato da $3$ gate \textsc{cnot}. 
\end{enumerate}
Ovvero questo circuito richiede di $\Theta(n^2)$ gate ed è, dunque, l'implementazione di un algoritmo $\Theta(n^2)$ per la \textit{trasformata di Fourier quantistica}.

In contrasto i migliori algoritmi classici per la computazione della trasformata di Fourier discreta (come, per esempio, \textsc{fft}) su $2^n$ elementi richiedono $\Theta(n2^n)$ gate per la propria implementazione, ovvero esponenzialmente più gate della \textsc{qft}. Superficialmente questo dato sembra miracoloso, vista l'importanza della trasformata di Fourier discreta nelle applicazioni pratiche.
Purtroppo l'algoritmo quantistico non può sostituire direttamente l'algoritmo classico, in quanto non è possibile ottenere direttamente i coefficienti delle immagini, se non perturbando il sistema.
Per sfruttare, quindi, le potenzialità di questo algoritmo bisogna agire in modo più sottile.

\subsection{Stima di fase}\label{sec:alg_phase}
La trasformata di Fourier è alla base dell'algoritmo per la risoluzione del seguente problema: dato un operatore unitario $U$ su $\C{m}$ e un suo autovettore $\ket{u}$ a cui è associato l'autovalore $e^{2 \pi i \varphi}$, individuare $\varphi$, la fase di $\ket{u}$.
Un algoritmo che risolve questo problema è detto algoritmo di \textit{stima di fase}.
Nell'esecuzione dell'algoritmo saranno chiamati varie volte dei gate $U^{2^j}$-controllati, ovvero i gate controllati che eseguono $2^j$ volte consecutive l'operatore $U$.
Ai fini dell'algoritmo di \textit{stima di fase} non è importante la realizzazione del gate $U$, che sarà considerato una \textit{scatola nera} o un \textit{oracolo} e non influenzerà il conto per la complessità computazionale dell'algoritmo.

\subsubsection{Circuito quantistico}
\begin{figure}[h]
\begin{center}
 \begin{quantikz}[column sep=1.5em, row sep=1.6em, thin lines]
     \lstick{$\ket{0}$}  & \gate{H^{\otimes{t}}} \qwbundle{t} \rstick{$\ \ket{j}$\\\vspace{2mm}} & \ctrl{1} & \gate{FT^\dag} & \meter{} & \cw\rstick{$m$}\\
     \lstick{$\ket{u}$} & \qwbundle[alternate]{} & \gate{U^j}\qwbundle[alternate]{} & \qwbundle[alternate]{} & \qwbundle[alternate]{} & \rstick{$\ket{u}$} \qwbundle[alternate]{}
 \end{quantikz}
\end{center}
\caption{Circuito che implementa l'algoritmo di stima di fase.}
\label{crq:QPF}
\end{figure}
L'algoritmo che verrà descritto in questa sezione fa uso di $2$ registri: un primo, di $t$ qubit, inizializzato allo stato $\ket{0}$, dove verrà salvata la stima della fase $\varphi$, e un secondo, inizializzato nello stato $\ket{u}$, abbastanza capiente da rappresentare l'autovettore di $U$.
La scelta di $t$, lunghezza del primo registro, si mostrerà dipendere dall'accuratezza con cui si vuole ottenere la stima della fase $\varphi$ e dalla probabilità di successo dell'algoritmo. 

L'algoritmo di \textit{stima di fase} è rappresentato nel circuito in figura \ref{crq:QPF}, la cui spiegazione nel dettaglio è a seguire:
Innanzitutto si opera con un gate di Hadamard su tutti i $t$ qubit del primo registro, che porta allo stato
\begin{equation}
 \ket{j} \otimes \ket{u} = \frac{\left(\ket{0} + \ket{1} \right) \left(\ket{0} + \ket{1} \right) \dots \left(\ket{0} + \ket{1} \right)}{2^{t/2}} \otimes \ket{u}.
\end{equation}
A questo punto, come descritto più in dettaglio nel circuito rappresentato in figura \ref{crq:QPF_inizio}, si opera con il gate controllato $U^j$, ove $j$ è lo stato del qubit nel primo registro.
Dal diagramma si desume che il gate controllato $U^j$ è implementato da successivi gate controllati $U^{2^k}$, i quali agiscono sempre sul qubit nello stato $\ket{u}$. 
In particolare il gate controllato $U$ agisce sul 2-qubit nello stato $\left(\ket{0} + \ket{1} \right) \ket{u}$ nel seguente modo
\begin{equation}
 CU \left(\ket{0} + \ket{1} \right) \ket{u} = \ket{0}\ket{u} + \ket{1} \left(e^{2 \pi i \varphi} \ket{u} \right) = \left(\ket{0} + e^{2 \pi i \varphi} \ket{1} \right) \ket{u}
\end{equation}
Analogamente agiscono i gate $U^{2^j}$ controllati, per cui, in seguito al circuito \ref{crq:QPF_inizio}, il primo registro si trova nello stato

\begin{equation}\label{eqn:QPF_half}
 \frac{\left(\ket{0} + e^{2 \pi i 2^{t-1} \varphi} \ket{1} \right) \cdots \left(\ket{0} + e^{2 \pi i 2^0 \varphi} \ket{1} \right)}{2^{t/2}} = \frac{1}{2^{t/2}} \sum_{k=0}^{2^t-1} e^{2 \pi i k \varphi} \ket{k},
\end{equation}
mentre il secondo per tutta la computazione è restato, e resterà, nello stato $\ket{u}$.
\begin{figure}
\begin{center}
\begin{quantikz}[column sep=1.3em, row sep=0.3em, thin lines]
     \lstick{$\ket{0}$} & \gate{H} & \qw & \qw & \qw & \ \ldots\ \qw & \ctrl{7} & \rstick{$\ket{0} + e^{2 \pi i \left(2^{t-1} \varphi \right)} \ket{1}$} \qw \\
     \lstick{$\vdots \ \ $} & & & & & {\vdots} & & & \rstick{$\ \vdots$} \\
     \lstick{$\ket{0}$} & \gate{H} & \qw & \ctrl{5} & \qw & \ \ldots\ \qw & \qw & \rstick{$\ket{0} + e^{2 \pi i \left(2^{1} \varphi \right)} \ket{1}$} \qw \\
     \lstick{$\ket{0}$} & \gate{H} & \ctrl{4} & \qw & \qw & \ \ldots\ \qw & \qw & \rstick{$\ket{0} + e^{2 \pi i \left(2^{0} \varphi \right)} \ket{1}$} \qw \\
     & & & & & & & & \\
     & & & & & {\vdots} & & & \\
     & & & & & & & & \\
     \lstick{$\ket{u}$} & \qwbundle[alternate]{} & \gate{U^{2^0}}\qwbundle[alternate]{} & \gate{U^{2^1}}\qwbundle[alternate]{} & \qwbundle[alternate]{} & \ \ldots\ \qwbundle[alternate]{} & \gate{U^{2^{t-1}}}\qwbundle[alternate]{} & \rstick{$\ket{u}$} \qwbundle[alternate]{} \\
     & & & & & & & & \\
 \end{quantikz}
\end{center}
\caption{Prima parte del circuito \ref{crq:QPF} in dettaglio}
\label{crq:QPF_inizio}
\end{figure}

La seconda parte della computazione consiste nell'applicare al primo registro la trasformata di Fourier inversa, per la quale un circuito può essere l'inverso del circuito \ref{crq:QFT} per la trasformata di Fourier, descritto nella sezione precedente.
Successivamente si misura tale registro nella base computazionale.

Per comprendere il funzionamento dell'algoritmo è utile supporre, inizialmente, che $\varphi$ sia un numero razionale, la cui scrittura $0.\varphi_1\varphi_2\dots\varphi_m$ abbia $m \leq t$ cifre e, quindi, sia rappresentabile nel primo registro. 
Chiamati $\varphi_{m+1} = \dots = \varphi_t = 0$, osi ottiene che lo stato \eqref{eqn:QPF_half} può essere riscritto come
\begin{equation}
 \frac{\left(\ket{0} + e^{2 \pi i 0.\varphi_t} \ket{1} \right) \left(\ket{0} + e^{2 \pi i 0.\varphi_{t-1}\varphi_t} \ket{1} \right) \cdots \left(\ket{0} + e^{2 \pi i 0.\varphi_1\varphi_2 \dots \varphi_t} \ket{1} \right)}{2^{t/2}}.
\end{equation}
Confrontato con la rappresentazione prodotto della trasformata di Fourier \eqref{lem:QFT_PR}, rende evidente il fatto che, in seguito alla trasformata di Fourier inversa, il primo registro si troverà nello stato $\ket{\varphi_1\varphi_2\dots\varphi_t}$. 
Da questo stato la misurazione nella base computazionale restituirà sempre il valore esatto di $\varphi$.

\subsubsection{Accuratezza e probabilità di successo}

Nel caso generale il processo è più complesso e non restituisce sempre il risultato richiesto, per cui richiede un'analisi più attenta.
Si chiami $0 \leq b \leq 2^t -1$ il massimo numero tale che $b/2^t = 0.b_1b_2\dots b_t \leq \varphi$, ovvero tale che $b/2^t$ è la miglior approssimazione per difetto di $\varphi$, lunga $t$ bit. 
Si definisce, inoltre, $\delta := \varphi - b/2^t$, il quale, chiaramente, soddisfa $0 \leq \delta \leq 2^{-t}$.
Lo stato \eqref{eqn:QPF_half}, in seguito alla trasformata di Fourier inversa, diventa
\begin{equation}
 \frac{1}{2^t} \sum_{k,l = 0}^{2^t-1} e^{-\frac{2 \pi i k l}{2^t}} e^{2 \pi i \varphi k} \ket{l},
\end{equation}
i cui coefficienti $\alpha_l$ dei termini $\ket{(b + l) \mod 2^t}$ sono
\begin{equation}
 \alpha_l = \frac{1}{2^t} \sum_{k=0}^{2^t-1} \left(e^{2 \pi i \left(\varphi - (b+l)/2^t\right)}\right)^k.
\end{equation}
Prendendo la somma parziale della serie geometrica possono essere riscritti come
\begin{equation}\label{eqn:dirichlet_kernel}
 \alpha_l = \frac{1}{2^t} \left( \frac{1 - e^{2 \pi i 2^t \left(\varphi - (b+l)/2^t\right)}}{1 - e^{2 \pi i \left(\varphi - (b+l)/2^t\right)}} \right) =
 \frac{1}{2^t} \left( \frac{1 - e^{2 \pi i \left(2^t\delta - l\right)}}{1 - e^{2 \pi i \left(\delta - l/2^t\right)}} \right).
\end{equation}
Per studiare l'errore commesso dal metodo si introduce un valore $e$, intero positivo, che rappresenta la tolleranza sul calcolo di $b$ che si è disposti ad accettare.
Tale valore $e$ rappresenta un errore di $e/2^t$ nel calcolo di $b/2^t$, che diventa di $e/2^t + \delta \leq (e+1)/2^t$, in quanto $0 \leq \delta \leq 2^{-t}$, per $\varphi$.
Vogliamo studiare, e limitare, la probabilità di ottenere, in seguito alla misurazione, un valore $m$ che non rispetti questa tolleranza, ovvero per cui $|m-b| > e$. 
La probabilità di osservare un tale $m$ è
\begin{equation}\label{eqn:QPF_prob_defn}
 \P\left( |m-b| > e \right) =
 \sum_{-2^{t-1}<l \leq -(e+1)} |\alpha_l|^2 + \sum_{e+1 \leq l \leq 2^{t-1}} |\alpha_l|^2
\end{equation}
Inoltre per ogni $\theta$ reale $1 - e^{i\theta} \leq 2$, da cui
\begin{equation}
 |\alpha_l| \leq \frac{2}{2^t \left| 1-e^{2 \pi i (\delta -l/2^t)} \right|}.
\end{equation}
Si dimostra, inoltre, che, se $-\pi \leq \theta \leq \pi$, allora $\left| 1-e^{i\theta}\right| \geq 2 |\theta|/\pi$. In particolare, se $-2^{t-1} < l \leq 2^{t-1}$, si ha $-\pi \leq 2\pi (\delta - l/2^t) \leq \pi$, da cui
\begin{equation}\label{eqn:QPF_magg-alpha_l}
 |\alpha_l| \leq \frac{2}{2^{t+1}(\delta - l/2^t)}.
\end{equation}
Sostituendo \eqref{eqn:QPF_magg-alpha_l} in \eqref{eqn:QPF_prob_defn}, si ottiene
\begin{equation}
 \P\left( |m-b| > e \right) \leq
 \frac{1}{4} \left[ \sum_{l=-2^{t-1} +1}^{-(e+1)} \frac{1}{(l - 2^t\delta)^2} + \sum_{l=e+1}^{2^{t-1}} \frac{1}{(l - 2^t\delta)^2} \right],
\end{equation}
che, in quanto $0 \leq 2^t \delta \leq 1$, diventa
\begin{align}
 \P\left( |m-b| > e \right) &\leq
 \frac{1}{4} \left[ \sum_{l=-2^{t-1} +1}^{-(e+1)} \frac{1}{l^2} + \sum_{l=e+1}^{2^{t-1}} \frac{1}{(l - 1)^2} \right]\\
 &\leq \frac{1}{2} \sum_{l=e}^{2^{t-1} - 1} \frac{1}{l^2}\\
 &\leq \int_{e-1}^{2^{t-1}-1} \frac{1}{l^2} \mathrm{d}l\\
 &\leq \frac{1}{2(e-1)}. \label{eqn:QPF_prob_err}
\end{align}
Per ottenere un'approssimazione di $\varphi$ accurata a $2^{-n}$ si sceglie ${e = 2^{t-n}-1}$, per quanto visto quando si è introdotta la tolleranza $e$. Inoltre si può scegliere $t$ arbitrariamente grande per ottenere con grande probabilità una stima che soddisfa tale requisito di accuratezza.
In particolare da \eqref{eqn:QPF_prob_err} si ricava che la probabilità  di ottenere un risultato entro i requisiti è $1 - \frac{1}{2(2^p-2)}$, posto $t = n + p$. 
Da questo risultato si desume che, per ottenere un avere corretto con probabilità $1 - \epsilon$, si deve scegliere
\begin{equation}
 t = n + \left\lceil \log_2 \left(2 + \frac{1}{2\epsilon} \right) \right\rceil.
\end{equation}
Un ultimo commento sull'algoritmo: nella descrizione precedente si è richiesto di preparare il secondo registro nello stato $\ket{u}$, autovettore dell'operatore $U$. Non sempre tale preparazione è possibile, ma ciò non preclude l'utilizzo di questo algoritmo.
È possibile, infatti, preparare una sovrapposizione di autovettori $\ket{\psi} = \sum_u c_u \ket{u}$.
Applicando l'algoritmo ai nuovi registri $\ket{0}\ket{\psi}$, per linearità delle operazioni coinvolte, restituirà come risultato $\sum_u c_u \ket{\widetilde{\varphi_u}} \ket{u}$, dove, con grande probabilità, $\widetilde{\varphi_u}$ è una buona approsimazione di $\varphi_u$.
Misurando il primo registro, quindi, si ottiene con probabilità $|c_u|^2$ una stima della fase di $\ket{u}$, autovettore di $U$.
Si è riusciti lo stesso a ottenere il risultato richiesto, al costo di introdurre una maggiore aleatorietà nell'output del programmma.

\begin{algo}[Stima di fase quantistica]\*
\begin{description}
  \item [Input:] (1) una scatola nera per eseguire il gate $U^j$ controllato, dove $j$ è un intero arbitrario, (2) un autovettore $\ket{u}$ di $U$, con autovalore $e^{2 \pi i \varphi_u}$ (o una combinazione lineare di tali autovettori), (3) $t = n + \left\lceil \log_2 \left(2 + \frac{1}{2\epsilon} \right) \right\rceil$ qubit inizializzati nello stato $\ket{0}$.
  \item [Output:] $\widetilde{\varphi_u}$, un'approsimazione di $n$ bit di $\varphi_u$.
  \item [Runtime:] $O(t^2)$ operazioni (dovute alla trasformata di Fourier inversa) e una chiamata alla scatola nera per l'esecuzione del gate $U^j$ controllato. Ritorna un output corretto con probabilità almeno $1 - \epsilon$.
  \item [Procedura:] \ 
    \begin{enumerate}
     \item ${\displaystyle\ket{0}\ket{u}}$
     \hfill {\small stato iniziale}
     \item ${\displaystyle\stackrel{H^{\otimes t}}{\longmapsto} \frac{1}{\sqrt{2^t}} \sum_{j=0}^{2^t-1} \ket{j}\ket{u}}$
     \hfill {\small sovrapposizione iniziale}
     \item ${\displaystyle\stackrel{CU^j}{\longmapsto} \frac{1}{\sqrt{2^t}} \sum_{j=0}^{2^t-1} \ket{j} U^j \ket{u}}$
     \hfill {\small scatola nera} \\*
     \ \ ${\displaystyle\quad\quad = \frac{1}{\sqrt{2^t}} \sum_{j=0}^{2^t-1} e^{2 \pi i j \varphi_u} \ket{j} \ket{u}}$
     \hfill {\small risultato scatola nera}
     \item ${\displaystyle\stackrel{FT^\dag}{\longmapsto} \ket{\widetilde{\varphi_u}} \ket{u}}$
     \hfill {\small risultato trasformata Fourier inversa}
     \item ${\displaystyle\longmapsto \widetilde{\varphi_u}}$
     \hfill {\small misura primo registro}
    \end{enumerate}
 \end{description}
\end{algo}

\subsection{Individuazione dell'ordine}\label{sec:alg_ordine}
Le prossime due sezioni sono strettamente correlate, tanto da risultare problemi equivalenti. 
Per una questione di chiarezza nell'esposizione si è preferito esporre i due argomenti in modo distinto, in particolare prima il problema di individuazione dell'ordine, per poi spiegarne l'applicazione per la fattorizzazione dei numeri interi.

La scelta di trattare questi algoritmi, inoltre, è dovuta al fatto che entrambi godono di un forte interesse sia teorico che applicativo. 
Dal punto di vista pratico possono essere usati per attaccare, in modo efficiente, la crittografia a chiave pubblica \textsc{rsa}, come descritto in appendice \ref{app:Crypto}. 
Da un punto di vista più teorico, invece, sono una prima evidenza della possibilità che i computer quantistici siano effettivamente più potenti di quelli classici.
Potrebbero, quindi, essere un controesempio alla tesi forte di Church-Turing, di grande importanza nell'impostazione della teoria della complessità computazionale.

Il problema da risolvere è il seguente.
Sia $\Zn{n}$ l'anello delle classi di resto modulo $n$ -- l'anello quoziente di $\Z$ sull'ideale $n\Z$.
Si indichi con $\left(\Zn{n}\right)^*$ l'insieme degli elementi invertibili dell'anello $\Zn{n}$, considerato con la naturale struttura di gruppo moltiplicativo.
Il problema di ricerca dell'ordine consiste, dato un arbitrario elemento $x$ del gruppo $\left(\Zn{n}\right)^*$, nel trovarne l'ordine.
In altre parole si chiede di trovare il minimo $r \in \N$ tale che $x^r = 1 \mod N$. 

\subsubsection{Circuito quantistico}
\begin{figure}[h]
\begin{center}
 \begin{quantikz}[column sep=1.5em, row sep=1.6em, thin lines]
     \lstick{$\ket{0}$}  & \gate{H^{\otimes{t}}} \qwbundle{t} \rstick{$\ \ket{j}$\\\vspace{2mm}} & \ctrl{1} & \gate{FT^\dag} & \meter{} & \cw\rstick{$\varphi$} \\
     \lstick{$\ket{1}$} & \qw\qwbundle{L} & \gate{x^j \mod n} & \qw & \qw & \rstick{$\ket{1}$} \qw
 \end{quantikz}
\end{center}
\caption{Circuito che implementa l'algoritmo di individuazione dell'ordine, si noti l'estrema somiglianza al circuito \ref{crq:QPF} per la stima di fase.}\label{crq:QOF}
\end{figure}
Per illustrare l'algoritmo di ricerca dell'ordine è necessario introdurre il seguente operatore unitario $U_{x,N}$:
\begin{equation}
 U_{x,N}\ket{y} := \ket{xy \mod N},
\end{equation}
dove $y \in \left\{0,1\right\}^L$, con $L:=\left\lceil \log_2(N) \right\rceil$, ovvero il minimo $L \in \N$ t.c. $N \leq 2^L$.
Si assume, inoltre, nella definizione di $U_{x,N}$ che, se $N \leq y \leq 2^L -1$, allora $xy \mod N$ sia ancora $y$, ovvero $U_{x,N}$ agisce in modo non banale solo sugli $0 \leq y \leq N$.
Definito questo operatore, e mostrato che è unitario, l'algoritmo di individuazione dell'ordine si ridurrà ad una stima di fase per l'operatore $U_{x,N}$.

Innanzitutto si dimostra che è un operatore unitario: 
\begin{lem}\label{lem:unitariety}
 $U_{x,N}:\ket{y} \mapsto \ket{xy \mod N}$ è un operatore unitario
\end{lem}
\begin{proof}
 $x$ è un elemento del gruppo degli invertibili di $\Zn{n}$, quindi la sua azione per moltiplicazione a sinistra su $(\Zn{n})^*$ può essere vista, per il teorema di Cayley, come una permutazione $\sigma$ sull'insieme finito $(\Zn{n})^*$.
 Tale permutazione può essere rappresentata sotto forma di matrice come
 \begin{equation}
  X_{ij} := \delta_{\sigma(j),j},
 \end{equation}
 dove $\delta$ è la delta di Kronecker.
 Questa matrice è chiaramente a coefficienti reali, per cui la sua aggiunta corrisponderà con la trasposta.
 In particolare $\sigma^{-1}$, l'inversa di $\sigma$ sarà la permutazione per cui $\sigma^{-1}\left(\sigma(j)\right) = j$, ovvero $\sigma^{-1}: \sigma(j) \mapsto j$.
 Ne segue che la rappresentazione matriciale di $\sigma^{-1}$ sarà 
 \begin{equation}
 Y_{i,j} := \delta_{j,\sigma(j)}.
 \end{equation}
 È chiaro che $Y$ è la trasposta di $X$, ovvero $X$ -- e di conseguenza $U_{x,N}$ -- è unitaria.
\end{proof}
Per riuscire ad applicare l'algoritmo di ricerca di fase, inoltre, è necessario individuare degli autovettori.
Un esempio sono, al variare di $0 \leq s \leq r-1$, gli stati definiti come
\begin{equation}\label{defn:eig_QOF}
 \ket{u_s} := \frac{1}{\sqrt{r}}\sum_{k=0}^{r-1}\exp\left(-\frac{2 \pi i }{r} sk \right)\ket{x^k \mod N}.
\end{equation}
Su di essi, infatti, $U_{x,N}$ agisce, per linearità, nel seguente modo
\begin{align}
 U_{x,N} \ket{u_s} &= \frac{1}{\sqrt{r}}\sum_{k=0}^{r-1}\exp\left(-\frac{2 \pi i }{r} sk \right)\ket{x^{k+1} \mod N}\\
 &= \exp\left(\frac{2 \pi i}{r} s \right)\ket{u_s}.
\end{align}
Da questi autovettori si può ottenere, con grande probabilità e accuratezza, una stima di $\exp\left(\frac{2 \pi i s}{r}\right)$, dalla quale, con buona probabilità -- se $s$ ed $r$ sono coprimi -- si può ricavare $r$.

Purtroppo per preparare lo stato $\ket{u_s}$, per un qualsiasi $s$, è necessario conoscere l'ordine $r$ di $x$, ovvero il risultato cercato.
Per ovviare questo problema si sfrutta il risultato ottenuto in \eqref{eqn:Sum_Rth_1_rad} riguardo alla somma delle radici $R$-esime dell'unità.
Esso, infatti, garantisce che $\forall k \in \left\{1, \dots, r-1 \right\}$
\begin{equation}
 \sum_{s=0}^{r-1} \exp\left(-\frac{2 \pi i}{r} sk\right) = 0.
\end{equation}
Mentre, per $k = 0$,
\begin{equation}
 \sum_{s=0}^{r-1} \exp\left(-\frac{2 \pi i}{r} s0\right) = r.
\end{equation}
Queste osservazioni comportano che, sommando su $s$ in \eqref{defn:eig_QOF}, rimane solo la somma con $k=0$, ovvero:
\begin{equation}
 \frac{1}{\sqrt{r}}\sum_{s=0}^{r-1} \ket{u_s} = \frac{1}{r}\sum_{s=0}^{r-1} \exp\left(-\frac{2 \pi i }{r} s0 \right)\ket{1} = \ket{1}.
\end{equation}
Da tutto ciò si desume che, applicando l'algoritmo di stima di fase per $U_{x,N}$, usando $t = 2L + 1 + \left\lceil \log_2\left(2 + \frac{1}{2\epsilon}\right)\right\rceil$ qubit nel primo registro e preparando il secondo registro nello stato $\ket{1}$ -- di facile preparazione -- si otterrà una stima $\varphi$ della fase  $s/r$, accurata a $2L+1$ bit, dove $s$ è scelto uniformemente in $0, \dots, r-1$.

A questo punto della computazione termina la fase prettamente quantistica e inizia un processo di elaborazione di $\varphi$ che sfrutta algoritmi classici e può, quindi, essere svolta su un modello di computazione classico.
Per questo motivo il circuito in figura \ref{crq:QOF} termina con la misurazione di $\varphi$ e non rappresenta le prossime operazioni.

In particolare, fissato $s$, si è a conoscenza del numero razionale $\varphi$, stima di $s/r$ accurata a $2L+1$ bit. 
Per ottenere $r$ da $\varphi$ si sfruttano l'algoritmo delle frazioni continue, che fornisce approssimazioni sempre più accurate di $\varphi$, descritto a seguire, e il teorema \ref{thm:Cont_frac}, la cui dimostrazione, per non appesantire troppo questa spiegazione, è stata relegata all'appendice \ref{app:Cont_frac}.
\begin{algo}[Espansione in frazione continua]\ \\
 Ogni numero razionale positivo può essere rappresentato come frazione continua finita ``semplice'', ovvero con un'espressione come
 \begin{equation}
  [a_0;a_1,a_2,\dots,a_n] := 
  a_0 + \cfrac{1}{a_1 
      + \cfrac{1}{a_2 
      + \cfrac{1}{\dots + \cfrac{1}{a_n} } } },
 \end{equation}
 in cui $a_0 \in \N$ e $a_1, \dots, a_n \in \N^+:=\left\{1,2,\dotsc\right\}$.
 In questa notazione si chiama convergente $j$-esimo, con $0 \leq j \leq n$, l'espansione $[a_0;a_1,a_2,\dots,a_j]$, corrispondente a una frazione $p_j/q_j$.
 Questa espansione si può trovare operando con il seguente
 \begin{description}
 \item[Algoritmo:] Sia $x = p_0/p_1$ un numero razionale. Si ricavano, iterativamente, $a_j$ operando nel seguente modo. $a_0 = \lfloor x \rfloor$ è la parte intera di $x$, per cui 
  \begin{equation}
   x = a_0 + \frac{p_2}{p_1} = a_0 +\frac{1}{\frac{p_1}{p_2}},
  \end{equation}
  in cui, nell'ultimo passaggio, si è invertita la frazione. In particolare, in quanto $a_0 = \lfloor x \rfloor$ segue che $p_2/p_1 < 1$, ovvero $p_2 < p_1$.
  Chiamato $x_1 := p_1/p_2$ si ripete su $x_1$ il processo appena svolto, ad ottenere $a_1$ e $x_2 := p_2/p_3$, con $p_3 < p_2$.
  Iterando si calcolano $a_n$, $x_{n+1}$ e $p_{n+2}$ a partire da $x_n$, costruendo un'espansione in frazione continua per $x$.
  Tale algoritmo si arresta quando $x_n = \lfloor x_n \rfloor$ è intero, ovvero quando $p_{n+1} = 0$.
  In particolare, per ogni $x$ razionale, $p_j$ è una successione a coefficienti in $\N^+$ e strettamente decrescente. Ne consegue che, in un numero finito di passi $p_j = 0$, ovvero l'algoritmo si ferma.
 \end{description}
\end{algo}
In appendice \ref{app:Cont_frac}, nel teorema \ref{thm:Cont_frac_convergents}, si definisce un algoritmo per il calcolo dei convergenti.
Questo algoritmo verrà usato, ma è importante citare alcune sue proprietà prima di usarlo.
Come mostrato nel lemma \ref{lem:Cont_frac_coprimality}, dimostrato in appendice, le frazioni costruite da tale algoritmo sono ridotte ai minimi termini.
Inoltre, come conseguenza dello stesso teorema \ref{thm:Cont_frac_convergents}, la successione dei denominatori delle frazioni corrispondenti ai convergenti è strettamente crescente.

Infine, prima di continuare, è necessario citare il seguente risultato:
\begin{thm}\label{thm:Cont_frac}
 Siano $\varphi$ ed $s/r$ razionali tali che
 \begin{equation}
  \left| \frac{s}{r} - \varphi \right| \leq \frac{1}{2r^2}
 \end{equation}
 Allora $s/r$ è un convergente della frazione continua di $\varphi$
\end{thm}
Nel caso analizzato $\varphi$ è un'approssimazione di $s/r$ precisa a $2L + 1$ bit, per cui $\left| s/r - \varphi \right| \leq 2^{-2L - 1} \leq \frac{1}{2r^2}$, in quanto $r < N \leq 2^L$.
Si può, quindi, applicare il teorema.
In quanto il valore $r$ è ignoto, mentre $N$ è noto e vale la più forte disuguaglianza $\left| s/r - \varphi \right| \leq \frac{1}{2N^2}$ è utile concentrarsi su questa.
In quanto $2N^2 > 2r$ si sa che in $\left(\varphi - \frac{1}{4N^2}, \varphi + \frac{1}{4N^2} \right)$ esiste un solo numero del tipo $a/r$, ovvero $s/r$.
Si consideri, ora, $m < r$, allora per ogni frazione $l/m$ si avrà, necessariamente
\begin{equation}
 \left| \frac{l}{m} -\frac{s}{r} \right| = \left| \frac{lr-sm}{rm} \right| > \frac{1}{rm} > \frac{1}{N^2}.
\end{equation}
Da cui $l/m$ non può distare meno di $1/2N^2$ da $\varphi$.
Ovvero, per ogni $m < r$, non esiste alcuna frazione con $m$ a denominatore nell'intervallo $\left(\varphi - \frac{1}{4N^2}, \varphi + \frac{1}{4N^2} \right)$.
La frazione con il più piccolo denominatore che vi appartiene è, quindi, $s/r$.
Nell'esecuzione dell'algoritmo per i convergenti della frazione continua, dunque, $s/r$ si può individuare come la prima frazione $l/m$ che soddisfa la condizione
\begin{equation}
  \left| \frac{l}{m} - \varphi \right| < \frac{1}{2N^2}.
\end{equation}

È, però, da notare che l'algoritmo delle frazioni continue genera denominatori e numeratori coprimi tra loro. Nel caso in cui, casualmente, MCD$(s,r) \neq 1$, allora, il denominatore ottenuto non sarà $r$, ma $r' \mid r$. 
Nella prossima sezione si analizzerà la probabilità di successo dell'algoritmo e verranno proposte delle strategie operative per mitigare il problema dovuto alla non coprimalità di $s$ ed $r$.

\subsubsection{Probabilità di successo}
Per finalizzare l'algoritmo è interessante studiare quali sono le cause di insuccesso di tale algoritmo e quali strategie possono essere implementate per mitigarle.
Si introdurranno queste soluzioni prima di passare allo studio del costo computazionale dell'algoritmo, per poter dare un risultato più accurato.

Il primo motivo per cui questo algoritmo potrebbe fallire è che, con probabilità minore di $\epsilon$ l'algoritmo di ricerca dell'ordine potrebbe restituire una stima non sufficientemente accurata di $s/r$. 
Questa probabilità di errore può essere resa piccola a piacere con un piccolo incremento delle dimensioni del circuito -- aumentando $t$.
Il secondo, e più importante, motivo di errore è dato, come accennato a fine della sezione precedente, dalla possibilità che $s$, scelto uniformemente tra $0$ er $r-1$, abbia fattori comuni con $r$. 
In tal caso il risultato dell'algoritmo di espansione in frazione continua restituirà come output $r' \mid r$, ma non $r$.
Si espongono, ora, 3 possibili strategie per mitigare il problema.

La prima consiste, semplicemente, nel notare che la probabilità che $s$ ed $r$ siano coprimi tra loro è, in realtà abbastanza elevata. Vale, infatti il seguente risultato, la cui dimostrazione è fuori dallo scopo di questa trattazione:
\begin{thm}\label{thm:Prime_extimate}
 Sia $\pi: \N \to \N$ la mappa che associa ad $n \in \N$ il numero di numeri primi minori o uguali di $n$. Vale la seguente disuguaglianza:
 \begin{equation}
  \pi(2n) \geq \frac{n}{\log(2n)}.
 \end{equation}
\end{thm}
Segue che almeno $r/r\log(r)$ dei naturali minori di $r$ sono primi. La probabilità che $s$ sia primo -- e quindi coprimo con $r$ -- è di $1/2\log(r) > 1/r\log(N)$, da cui, ripetendo l'algoritmo $2\log(N)$ volte si ha una grande probabilità di successo.

Un secondo metodo, più interessante, richiede di notare che, se $s \neq 0$, $r' \neq r$ è un fattore proprio di $r$. 
L'ordine di $x' := x^{r'} \mod N$ sarà, quindi, $r/r'$ e si potrà operare con lo stesso algoritmo su $x'$. 
È possibile che anche in questo caso $s'$ ed $r'$ non siano coprimi, richiedendo di procedere ulteriormente. 
Ad ogni passo, però, l'ordine $r^{(n)}$ di $x^{(n)}$ viene diviso per un fattore strettamente maggiore di 2. 
È necessario, allora, operare al più $\log_2(r) = O(L)$ volte per trovare l'ordine dell'ultima iterazione.
A quel punto, moltiplicando tra loro tutti i risultati ottenuti, si ottiene l'ordine di $x$.

L'ultimo metodo, invece, richiede solamente un numero finito di tentativi, a differenza dei primi due. 
Operando 2 volte si ottengono i risultati $r_1, s_1$ ed $r_2, s_2$.
Da questi, se $s_1$ ed $s_2$ non hanno fattori comuni, si può ricavare $r$ semplicemente prendendo il minimo comune multiplo di $r_1$ ed $r_2$.
È necessario, quindi, stimare la probabilità che $s_1$ ed $s_2$ non abbiano fattori comuni.
Per fare ciò si calcola la probabilità che ne abbiano, ovvero la probabilità dell'evento complementare.
Questa può essere espressa come la probabilità che esista almeno un primo $p$ che divide entrambi.
La probabilità da stimare è, quindi
\begin{equation}\label{eqn:QOF_3rdMethod_prob}
 P := 1 - \P\left(\exists\, p \text{ primo} : p \mid s_1 \text{ e } p \mid s_2\right).
\end{equation}
Per subadditività della probabilità, inoltre, si ottiene la maggiorazione
\begin{equation}
  \P\left(\exists\, p \text{ primo} : p \mid s_1 \text{ e } p \mid s_2\right) \leq \sum_{p}  \P\left(p \mid s_1 \text{ e } p \mid s_2\right),
\end{equation}
dove la somma è su tutti i numeri primi $p$.
In particolare i numeri $s_1$ ed $s_2$ sono scelti in modo indipendente l'uno dall'altro, per cui vale anche
\begin{equation}
  \P\left(p \mid s_1 \text{ e } p \mid s_2\right) = \P\left(p \mid s_1 \right) \P\left(p \mid s_2\right).
\end{equation}
Si deve, inoltre, notare che $s_1$ -- ma la stessa osservazione vale anche per $s_2$ -- è scelto uniformemente in $0, \dots, r-1$.
La probabilità che sia divisibile per $p$ è, quindi, data da
\begin{equation}
 \frac{\left|p\N \cap \left\{0, \dots, r-1\right\}\right|}{\left|\left\{0, \dots, r-1\right\}\right|} = \frac{\lfloor r/p \rfloor + 1}{r} \leq \frac{r/p + 1}{r} = \frac{1}{p} + \frac{1}{r} < \frac{1}{p}.
\end{equation}
Sostituendo queste maggiorazioni in \eqref{eqn:QOF_3rdMethod_prob} si ottiene la stima
\begin{equation}\label{eqn:QOF_3rdMethod_stima_parz}
 P = 1 - \P\left(\exists\, p \text{ primo} : p \mid s_1 \text{ e } p \mid s_2\right) > 1 - \sum_{p} \frac{1}{p^2},
\end{equation}
dove, come prima, la somma è sui numeri primi $p$.
Rimane, ora, solamente da stimare 
\begin{equation}\label{eqn:QOF_3rdMethod_stima_ultima}
 \sum_{p} \frac{1}{p^2} < \sum_{n=2}^\infty \frac{1}{n^2}.
\end{equation}
A tal fine si sfrutta il seguente risultato
\begin{equation}\label{eqn:QOF_3rdMethod_stima_magg}
 \int_{x}^{x+1} \frac{1}{y^2} \text{ d}y \geq \frac{2}{3x^2} \quad \forall x \geq 2.
\end{equation}
Ciò è vero in quanto $\int_{x}^{x+1} 1/y^2 \text{ d}y = 1/(x+x^2)$ e, inoltre, vale la seguente
\begin{align}
 \frac{1}{x+x^2} \geq \frac{2}{3x^2} \iff& 3x^2 \geq 2x^2 + 2\\
 \iff& x^2 - 2x \geq 0.
\end{align}
L'ultima disuguaglianza è chiaramente verificata per ogni $x \geq 2$.
Sfruttando il risultato di \eqref{eqn:QOF_3rdMethod_stima_magg} in \eqref{eqn:QOF_3rdMethod_stima_ultima} si arriva a
\begin{equation}
 \sum_{n=2}^\infty \frac{1}{n^2} \leq \frac{3}{2}\int_{2}^{\infty} \frac{1}{y^2} \text{ d}y = \frac{3}{4}.
\end{equation}
Sostituito in \eqref{eqn:QOF_3rdMethod_stima_parz} si ottiene la stima cercata, ovvero:
\begin{equation}
 P > 1 - \sum_{p} \frac{1}{p^2} > \frac{1}{4}.
\end{equation}
Ripetendo l'algoritmo 4 volte si ha, quindi, una grande probabilità di ottenere il risultato corretto.
La cosa importante da notare di questa strategia è che la probabilità di ottenere il risultato giusto non dipende da $N$.
Per $N$ grandi risulta, quindi, la scelta migliore, richiedendo solo un numero costante di ripetizioni.

\subsubsection{Costo computazionale}
Per eseguire il circuito \ref{crq:QOF}, ed analizzarne il costo computazionale, è necessario implementare un circuito che operi la seguente trasformazione
\begin{equation}
 \ket{z}\ket{y} \mapsto \ket{z}\ket{x^zy \mod N}
\end{equation}
A tal fine si userà, ed analizzerà, l'algoritmo di \textit{esponenziazione modulare}. 
Non verrà, però, trattato con estrema profondità, per la quale si rimanda a \cite{Article:Shor:1997}.

\begin{algo}[esponenziazione modulare]\label{algo:modular_exponentiation}\ \\
 L'algoritmo di \textit{esponenziazione modulare} opera nel seguente modo su una coppia di registri, il primo di lunghezza $t = 2L + 1 + \left\lceil \log_2\left(2 + \frac{1}{2\epsilon}\right)\right\rceil$ e il secondo di lunghezza $L = \lceil\log_2(N)\rceil$:
 \begin{equation}
  \ket{z}\ket{y} \mapsto \ket{z}\ket{x^zy \mod N}.
 \end{equation}
 Espresso $z$ in base binaria $z = z_t\dots z_1$, dove $z_1, \dots, z_t$ sono le cifre binarie di $z$, l'azione dell'algoritmo è analoga a:
 \begin{align}
  \ket{z}\ket{y} \mapsto& \ket{z}\ket{x^{z_t2^{t-1}}\cdot \dots \cdot x^{z_1 2^0} y \mod N}\\
  =& \ket{z} U^{z_t2^{t-1}} \dots U^{z_1 2^0}\ket{y}.
 \end{align}
 Nell'ultima riga si è espressa l'applicazione di una sequenza di gate $U_{x,N}^{2^j}$ controllati.
 Ciascuno di questi gate può essere implementato usando un registro ausiliario in cui calcolare in modo reversibile la funzione $x^z \mod N$.
 Si moltiplica, poi, sempre in modo reversibile il secondo registro, nello stato $\ket{y}$ per il contenuto del terzo registro.
 Il gate termina cancellando il contenuto del terzo registro eseguendo le operazioni usate per calcolare $x^z \mod N$ in ordine inverso.
 Come mostrato da \cite{Article:Shor:1997}, nella sezione 3, tale algoritmo, se implementato con l'usale algoritmo di prodotto di numeri interi (quello insegnato alle elementari) ha un costo di $O(L^2)$ gate.
 
 L'algoritmo di esponenziazione modulare, quindi, si implementa nelle seguenti due fasi.
 In primo luogo si calcolano, usando la procedura appena delineata, i valori $x^{2^j} \mod N$, per $0 \leq j \leq t-1$, elevando iterativamente al quadrato il valore precedente. Ciascun elevamento al quadrato ha un costo di $O(L^2)$ gate, da ripetere un totale di $t-1 = 2L +  \left\lceil \log_2\left(2 + \frac{1}{2\epsilon}\right)\right\rceil = O(L)$ di volte. Questa fase, quindi, ha un costo complessivo di $O(L^3)$.
 
 La seconda fase, invece, richiede di calcolare, nel secondo registro,
 \begin{equation}
  x^z \mod N = \left(x^{z_t2^{t-1}} \mod N\right)\cdot \dots \cdot \left(x^{z_1 2^0} \mod N \right),
 \end{equation}
 ovvero di eseguite i $t-1$ gate controllati $U_{x,N}^{2^j}$.
 Ciascuna di queste moltiplicazioni, come mostrato prima, richiede $O(L^2)$ operazioni per essere eseguita. Anche la seconda fase ha, quindi, un costo computazionale di $O(L^3)$ gate.
 
 Seguendo questa descrizione si può implementare un gate quantistico che opera la seguente trasformazione
 \begin{equation}
 \ket{z}\ket{y} \mapsto \ket{z}\ket{x^zy \mod N}
 \end{equation}
 a un costo di $O(L^3)$ gate. 
 Questo non è il risultato (asintotico) più efficiente conosciuto, per approfondire si può consultare l'articolo precedentemente citato.
\end{algo}

Per quanto riguarda l'algoritmo di espansione in frazione continua la richiesta computazionale è analoga. 
Sia, infatti, $x = p/q$ un numero razionale maggiore di 1, dove $p$ e $q$ sono coprimi.
Siano, inoltre, $a_0, \dots, a_N$ i primi $N$ termini dell'espansione in frazione continua semplice di $x$.
Prese $p_n$ e $q_n$ come definite nel teorema \ref{thm:Cont_frac_convergents} si ha che $p_n \geq 2p_{n-2}$ e $q_n \geq 2q_{n-2}$, come mostrato in \eqref{eqn:Cont_frac_convergents_increasing}.
Segue che $p_n, q_n \geq 2^{\lfloor n/2 \rfloor}$, per cui $2^{\lfloor N/2 \rfloor} \leq q \leq p$. 
Da ciò si ottiene che il numero di termini dell'espansione in frazione continua è $n = O(\log_2(p))$.
In particolare, se $p$ e $q$ sono numeri rappresentabili con $L$ bit, allora $n = O(L)$.
L'algoritmo per le frazioni continue, quindi, opera con $O(L)$ inversioni e sottrazioni, ciascuna di esse eseguibili con $O(L^2)$ operazioni.
Il suo costo computazionale complessivo è, come promesso, $O(L^3)$ gate.

Gli altri due algoritmi che compaiono nel circuito \ref{crq:QOF} sono il gate di Hadamard, con una richiesta di $O(L)$ operazioni e la trasformata di Fourier inversa, che ne richiede $O(L^2)$.

Infine, supponendo di adottare il terzo metodo per mitigare la possibilità di errore dovuta alla non coprimalità di $s$ ed $r$, si deve ripetere un numero costante di volte l'algoritmo.
Combinando tutti questi risultati si ottiene che il costo computazionale complessivo è di $O(L^3)$.

\begin{algo}[Individuazione dell'ordine quantistico]\*
\begin{description}
  \item [Input:] (1) $x, N \in \N$ t.c. MCD$(x,N) \neq 1$, (2) $t = 2L + 1 + {\left\lceil \log_2 \left(2 + \frac{1}{2\epsilon} \right) \right\rceil}$ qubit inizializzati nello stato $\ket{0}$, (3) $L$ qubit inizializzati nello stato $\ket{1}$.
  \item [Output:] il minimo $r \in \N$ tale che $x^r = 1 \mod N$.
  \item [Runtime:] $O(L^3)$ operazioni (per il calcolo dell'esponenziazione modulare). Ritorna un output corretto con probabilità $O(1)$.
  \newpage
  \item [Procedura:] \ 
    \begin{enumerate}
     \item ${\displaystyle\ket{0}\ket{1}}$
     \hfill {\small stato iniziale}
     \item ${\displaystyle\stackrel{H^{\otimes t}}{\longmapsto} \frac{1}{\sqrt{2^t}} \sum_{j=0}^{2^t-1} \ket{j}\ket{1}}$
     \hfill {\small sovrapposizione iniziale}
     \item ${\displaystyle\stackrel{U_{x,N}}{\longmapsto} \frac{1}{\sqrt{2^t}} \sum_{j=0}^{2^t-1} \ket{j} \ket{x^j \mod N}}$
     \hfill {\small risultato $U_{x,N}$}\\*
     \ \ ${\displaystyle\quad \approx \frac{1}{\sqrt{r2^t}} \sum_{s=0}^{r-1}\sum_{j=0}^{2^t-1} e^{2 \pi i s j / r} \ket{j} \ket{u_s}}$
     \item ${\displaystyle\stackrel{FT^\dag}{\longmapsto} \frac{1}{\sqrt{r}}\sum_{s=0}^{r-1} \widetilde{\ket{s/r}}\ket{u_s}}$
     \hfill {\small risultato trasformata Fourier inversa}
     \item ${\displaystyle\longmapsto \widetilde{s/r}}$
     \hfill {\small misura primo registro}
     \item ${\displaystyle\longmapsto s}$
     \hfill {\small algoritmo frazioni continue}
    \end{enumerate}
 \end{description}
\end{algo}

\subsection{Fattorizzazione degli interi}\label{sec:factoring}
Nelle due sezioni a seguire saranno descritti due problemi di teoria dei numeri di grande importanza teorica.
Questi sono il problema di fattorizzazione e il problema del logaritmo discreto, per la risoluzione dei quali non si dispone di un algoritmo classico efficiente.
In realtà si crede che questi problemi non ammettano una risoluzione classica efficiente e, per questo motivo, sono stati implementati vari protocolli crittografici basati su essi.

Si introdurranno, però, algoritmi quantistici efficienti per la risoluzione dei problemi di fattorizzazione e di calcolo del logaritmo discreto.
I motivi di interesse in questi algoritmi sono sostanzialmente due.
Innanzitutto, se si dimostrasse che questi problemi, classicamente, non ammettono risoluzione efficiente, si avrebbe una conferma che la computazione quantistica rappresenta un modello più potente di quello classico: un controesempio alla tesi forte di Church-Turing.
Inoltre le applicazioni pratiche, nel rendere inefficaci i sistemi crittografici citati -- come mostrato più in dettaglio in appendice \ref{app:Crypto} -- creano un forte interesse nell'utilizzo di questi algoritmi.

In questa sezione si ottiene una riduzione del problema di ricerca di un fattore primo al problema di individuazione dell'ordine, appena studiato.
Ottenuto questo risultato si ricava, di conseguenza, un algoritmo efficiente per la fattorizzazione dei numeri interi, come preannunciato.

La seguente riduzione assume che il numero $N$ di cui cercare la fattorizzazione sia dispari e non sia una potenza di un primo.
Questa restrizione non è molto pesante.
Il primo caso, infatti, può essere scartato dividendo ripetutamente per $2$ e, in notazione binaria, riconscere un numero pari non pone un problema.
Il secondo caso, invece, richiede di operare con il seguente algoritmo classico:
\begin{enumerate}
 \item Con un algoritmo veloce, per esempio quello descritto in \cite{Article:PerfectPowers}, si controlla se il numero $N$ è una potenza perfetta, ovvero la $b$-esima potenza di un altro numero naturale.
 \item Si controlla con un algoritmo veloce, si veda \cite{Article:PRIMESinP}, che tale $b$-esima radice di $N$ sia un numero primo.
\end{enumerate}

Per terminare la riduzione del problema di fattorizzazione alla ricerca dell'ordine sono, infine, necessari i seguenti risultati:
\begin{thm}
 Sia $N$ un numero di $L$ bit composito, ovvero non primo. 
 Sia $x$ una soluzione non banale, ovvero $1 < x < N-1$, dell'equazione $x^2 = 1 \mod N$.
 Allora almeno uno tra MCD$(x-1,N)$ e MCD$(x+1,N)$ è un fattore proprio, non banale, di $N$.
 In particolare la richiesta computazionale per il suo calcolo è di $O(L^3)$ operazioni.
\end{thm}
\begin{proof}
 $x^2 = 1 \mod N$ implica che $N$ divide $x^2 - 1 = (x+1)(x-1)$, ovvero $N$ ha almeno un fattore comune con $x-1$ o con $x+1$.
 Per ipotesi $1 < x < N-1$ comporta che $x-1 < x+1 < N$, ovvero il fattore comune non può essere $N$ stesso ed è, quindi, un fattore proprio di $N$.
 Per finire la dimostrazione basta notare che MCD$(x-1,N)$ e MCD$(x+1,N)$ si possono calcolare operando con l'algoritmo di Euclide, con una spesa di $O(L^3)$ operazioni -- come mostrato in \cite{Book:Knuth_2}.
\end{proof}

Il teorema appena dimostrato afferma che, individuato un particolare elemento di $\Zn{N}^*$, da esso si può trovare efficientemente un fattore proprio di $N$. 
Il problema si riconduce, ora, alla ricerca di tali elementi in $\Zn{N}^*$.
Per trovarli è sufficiente individuare degli elementi $y \in \Zn{N}^*$, di ordine $r$ pari, per cui $y^{r/2} \neq \pm 1 \mod N$.
Posto $x=y^{r/2}$ si ha una soluzione non banale di $x^2 = 1 \mod N$.
A priori potrebbe essere molto raro individuare tali elementi, fortunatamente non è il caso. 
L'obiettivo dei seguenti teoremi è di calcolare la probabilità che, scelto uniformemente un elemento in $\left(\Zn{N}\right) \setminus \{0\}$, soddisfi le proprietà appena enunciate.

\begin{lem}\label{lem:Prob_factorization}
 Siano $p$ un primo dispari ed $\alpha$ un numero naturale.
 Sia $2^d$ la più grande potenza di $2$ che divide $\varphi\left(p^\alpha\right)$.
 Allora, scelto $x \in \Zn{p^\alpha}^*$ uniformemente, $2^d$ divide l'ordine di $x$ con probabilità esattamente $1/2$.
\end{lem}
\begin{proof}
 In quanto $p$ è dispari $\varphi\left(p^\alpha\right) = p^{\alpha-1}(p-1)$ è pari, quindi $d \geq 1$.
 Si può mostrare -- si veda \cite{Book:Jacobson}, Teorema 4.19, pagina 274 -- che $\Zn{p^\alpha}^*$ è un gruppo ciclico.
 Sia $g$ un suo generatore e si consideri il generico elemento $g^k$, con $1 \leq k \leq \varphi\left(p^\alpha\right)$.
 Chiamato $r$ l'ordine di $g^k$ si considerano i due possibili casi per $k$.
 Se $k$ è dispari, in quanto $kr = \varphi\left(p^\alpha\right)$ e $2^d \mid \varphi\left(p^\alpha\right)$, allora $2^d \mid r$.
 Altrimenti $k$ è pari, quindi
 \begin{equation}
  g^{k\varphi\left(p^\alpha\right)/2} = \left(g^{\varphi\left(p^\alpha\right)}\right)^{k/2} = 1^{k/2} = 1_{\Zn{p^\alpha}^*}.
 \end{equation}
 Ovvero $r$, ordine di $g^k$, divide $\varphi\left(p^\alpha\right)/2$, per cui $2^d$ non divide $r$.
 
 Ricapitolando, $\Zn{p^\alpha}^*$ si divide in 2 insiemi di uguale cardinalità: gli elementi che possono essere scritti come $g^k$, con $k$ dispari, e quelli con $k$ pari. 
 Per i primi $2^d$ divide $r$, per gli ultimi no.
 Scegliendo uniformemente $x$ in $\Zn{p^\alpha}^*$ si ha, quindi, esattamente probabilità $1/2$ di trovare un elemento a cui $2^d$ divide l'ordine.
\end{proof}

Prima di calcolare esplicitamente la probabilità richiesta, si ricorda il seguente teorema.
Non verrà data una dimostrazione, che si può trovare in \cite{Book:QCQI}.
\begin{thm}[cinese dei resti]\label{thm:Chinese}
 Siano $m_1, \dots, m_n$ interi positivi, a due a due coprimi.
 Il sistema di equazioni
 \begin{equation}
  \begin{cases}
   x = a_1 \mod m_1\\
   x = a_2 \mod m_2\\
   \ \ \ \vdots\\
   x = a_n \mod m_n
  \end{cases}
 \end{equation}
 ammette soluzioni. In particolare tutte le soluzioni di tale sistema coincidono, modulo $M := m_1m_2 \dots m_n$
\end{thm}

\begin{thm}
 Sia $N = p_1^{\alpha_1} \dots p_m^{\alpha_m}$ la fattorizzazione di un numero composito dispari. 
 Sia $x$ un numero scelto uniformemente in $\left\{1,\dots,N-1\right\}$ coprimo con $N$. 
 Sia  $r$ l'ordine di $x$ modulo $N$.
 Allora
 \begin{equation}
  \P\left( 2 \mid r \text{ e } x^{r/2} \neq -1 \mod N \right) \geq 1 - \frac{1}{2^{m-1}}.
 \end{equation}
\end{thm}
\begin{proof}
 Per dimostrare la tesi verrà mostrato l'equivalente:
 \begin{equation}
  \P\left( 2 \nmid r \text{ o } x^{r/2} \neq -1 \mod N \right) \leq \frac{1}{2^{m-1}}.
 \end{equation}
 Per il teorema cinese dei resti scegliere $x$ uniformemente in $\Zn{N}^*$ è equivalente a scegliere uniformemente $x_j$ indipendentemente in $\Zn{p_j^{\alpha_j}}^*$ e a richiedere che $x = x_j \mod p_j^{\alpha_j}$ per ogni $j$.
 Si chiami $r_j$ l'ordine di $x_j$ in $\Zn{p_j^{\alpha_j}}^*$.
 Siano $2^{d_j}$ la più grande potenza di 2 che divide $r_j$ e $2^d$ la più grande che divide $r$.
 Se $r$ è dispari, in quanto, per ogni $j$, $r_j \mid r$, si ha che $d_j = 0$ per ogni $j$, in particolare tutti i $d_j$ coincidono.
 Se, invece, $x^{r/2} = -1 \mod N$, allora $x^{r/2} = -1 \mod p_j^{\alpha_j}$ per ogni $j$.
 Segue che non esiste alcun $j$ per cui $r_j \mid (r/2)$.
 Si sa, per costruzione, che $r_j \mid r$, per cui $d_j = d$ per ogni $j$, in particolare tutti i $d_j$ coincidono.
 Quest'analisi mostra che vale la seguente inclusione di eventi
 \begin{equation}\label{eqn:Factoring_thm_condition}
  \left\{2 \nmid r \text{ o } x^{r/2} \neq -1 \mod N\right\} \subseteq \left\{\exists d \in \N : d_j = d\ \ \forall\, 1 \leq j \leq m \right\},
 \end{equation}
 che, per monotonia della probabilità, comporta che
 \begin{equation}
  \P\left( 2 \nmid r \text{ o } x^{r/2} \neq -1 \mod N \right) \leq \P\left( \exists d \in \N : d_j = d\ \ \forall\, 1 \leq j \leq m \right).
 \end{equation}
 Per il lemma \ref{lem:Prob_factorization} si sa che, fissato $j$, la probabilità che $d_j = d$ è al più $1/2$ -- nel caso in cui $d$ è la massima potenza di $2$ che divide $\varphi\left(p^{\alpha_j}\right)$.
 Si nota, inoltre, che, per l'abitrarietà di $d$, si può imporre $d:=d_1$ e verificare la condizione in \eqref{eqn:Factoring_thm_condition} solo per $j \geq 2$.
 In quanto gli $x_j$ sono scelti indipendentemente, infine, la probabilità si fattorizza, ovvero:
 \begin{equation}
  \P\left(d_j = d_1\ \ \forall\, 2 \leq j \leq m \right) = \prod_{j=2}^m \P\left(d_j = d_1 \right) \leq \prod_{j=2}^m \frac{1}{2} = \frac{1}{2^{m-1}}.
 \end{equation}
\end{proof}
Operando come descritto in precedenza, quindi, si ha una grande probabilità di calcolare un fattore proprio di $N$ numero composito dispari. 
La probabilità di successo non dipende direttamente da $N$, ma dal numero di fattori in cui si decompone, ovvero non è del tipo $O\left(f(N)\right)$ per una funzione di $N$, ma è $O(1)$ e, in particolare, è sempre maggiore di $3/4$.

Riguardo all'efficienza è interessante notare come tutti i passaggi di questo algoritmo, a meno della ricerca dell'ordine, possano essere eseguiti efficientemente anche su un computer classico.
Questa riduzione è, quindi, una riduzione alla sola ricerca dell'ordine, in quanto l'efficienza dell'algoritmo di fattorizzazione dipende principalmente dalla velocità di quest'ultimo.

È, ora, possibile dare una descrizione sintetica della riduzione:
\begin{algo}[Riduzione della fattorizzazione alla ricerca dell'ordine]\
 \begin{description}
  \item [Input:] un numero $N$.
  \item [Output:] un fattore non banale di $N$.
  \item [Runtime:] $O\left((\log_2 N)^3\right)$ operazioni. Ritorna un output corretto con probabilità $O(1)$.
  \item [Procedura:] \ 
    \begin{enumerate}
     \item Se $N$ è pari ritorna il fattore $2$.
     \item Se $N = a^b$ per $a \geq 1$ e $b \geq 2$ -- individuato con l'algoritmo classico descritto in \cite{Article:PerfectPowers} -- restituisce $a$.     
     \item Sceglie uniformemente $1 \leq x \leq N-1$. Se MCD$(x,N) > 1$ ritorna il fattore non banale MCD$(x,N)$.
     \item Trova, con l'algoritmo di individuazione dell'ordine, l'ordine $r$ di $x$. Se $r$ non è pari l'algoritmo si ferma senza restituire un risultato.
     \item Se $r$ è pari e $x^{r/2} \neq \pm 1 \mod N$ calcola MCD$(x^{r/2}-1,N)$ e MCD$(x^{r/2}+1,N)$. Ritorna il fattore proprio di $N$ individuato tra i numeri appena calcolati. Se $x^{r/2} = \pm 1 \mod N$ l'algoritmo si ferma senza dare un risultato.
    \end{enumerate}
 \end{description}
\end{algo}
Per concludere propriamente la fattorizzazione è necessario iterare la ricerca di fattori propri, fino a che non si ha una completa fattorizzazione.
In particolare se, al punto $3.$ si sceglie un elemento $x$ il cui ordine è dispari o tale che $x^{r/2} = \pm 1 \mod N$ si ripete il punto $3.$, scegliendo un elemento diverso.
Se, invece, restituisce un fattore proprio $D$ si ripete l'algoritmo sul fattore $D$ e sul rapporto $N/D$.
Se gli elementi appena descritti sono numeri primi il punto $2.$ lo individua e si può fermare l'iterazione.

\subsection{Logaritmo discreto}
Il problema del calcolo del logaritmo discreto è il seguente.
Dato $p \in \N$ primo si sa che $\left(\Zn{p}\right)^*$ è un gruppo moltiplicativo ciclico.
Sia $a$ un generatore di tale gruppo e $b$ un qualsiasi altro elemento.
Il problema del logaritmo discreto chiede di individuare la più piccola potenza $a^s$ che soddisfi l'equazione $a^s = b \mod p$.

Per eseguire questo calcolo si descrive un algoritmo che usa tre registri.
Il terzo è di lunghezza $L := \lceil\log_2{p}\rceil$, ovvero la minima potenza di $2$ necessaria per rappresentare tutti gli elementi di $\left(\Zn{p}\right)^*$.
I primi due, invece, sono di lunghezza $t = 2L + 1 + \left\lceil \log_2\left(2 + \frac{1}{2\epsilon}\right)\right\rceil$, valore su cui dipende la probabilità di successo dell'algoritmo.

Infine, per rendere più chiara la generalizzazione di questo problema a gruppi arbitrari, si chiamerà $r$ l'ordine di $a$, invece di riferirsi ad esso con il suo valore esplicito: $p-1$.

\subsubsection{Circuito Quantistico}
\begin{figure}[h]
 \begin{center}
  \begin{quantikz}[column sep=1.8em, row sep=2.0em, thin lines]
   \lstick{$\ket{0}$} & \gate{H^{\otimes{t}}} \qwbundle{t} \rstick{$\ \ket{j}$\\\vspace{2mm}} & \slice{$\ket{\psi_1}$}\qw & \gate[wires=3][2.5cm]{U_f}\gateinput{$x$}\gateoutput{$x$}\slice{$\ket{\psi_2}$} & \gate{FT^\dag}\slice{$\ket{\psi_3}$} & \meter{} & \cw\rstick{$\varphi$} \\
   \lstick{$\ket{0}$} & \gate{H^{\otimes{t}}} \qwbundle{t} \rstick{$\ \ket{j}$\\\vspace{2mm}} & \qw & \gateinput{$y$}\gateoutput{$y$} & \gate{FT^\dag} & \meter{} & \cw\rstick{$\psi$} \\
   \lstick{$\ket{0}$} & \qw\qwbundle{L} & \qw & \gateinput{$0$}\gateoutput{$a^xb^y$} & \qw & \qw & \rstick{$\ket{v}$} \qw
  \end{quantikz}
 \end{center}
 \caption{Circuito che implementa l'algoritmo di calcolo del logaritmo discreto.}\label{crq:DLP}
\end{figure}

Per iniziare si definisce la funzione $f(x,y) := a^xb^y = a^{x+sy} \mod p$.
A tale funzione, come descritto nella sezione \ref{sec:Quantum_par} riguardo al parallelismo quantistico, si associa il gate quantistico $U_f$, che agisce come:

\begin{equation}
 \ket{x}\ket{y}\ket{z} \stackrel{U_f}{\longmapsto} \ket{x}\ket{y}\ket{z + f(x,y) \mod 2^L}.
\end{equation}

Il circuito \ref{crq:DLP} inizia creando, con i gate Hadamard, la sovrapposizione uniforme sui primi due qubit, ovvero lo stato
\begin{equation}
 \ket{\psi_1} = \frac{1}{2^t}\sum_{x,y = 0}^{2^t-1} \ket{x}\ket{y}\ket{0}.
\end{equation}
Operando con il gate $U_f$ su $\ket{\psi_1}$, in quanto il terzo qubit è nello stato $\ket{0}$, si arriva a
\begin{equation}
 \ket{\psi_2} = \frac{1}{2^t}\sum_{x,y = 0}^{2^t-1} \ket{x}\ket{y}\ket{a^xb^y \mod p} = 
 \frac{1}{2^t}\sum_{x,y = 0}^{2^t-1} \ket{x,y,a^{x+sy} \mod p}.
\end{equation}
Si definiscono, ora, in modo analogo alla sezione \ref{sec:alg_ordine} per l'individuazione dell'ordine, gli stati $u_k$
\begin{equation}
 \ket{u_k} := \frac{1}{\sqrt{r}}\sum_{l=0}^{r-1} \exp(-\frac{2 \pi i}{r} kl) \ket{a^k \mod p}.
\end{equation}
Si definisce, inoltre, per come opera sugli stati $\ket{j}$ della base computazionale, l'operatore $U_a$ unitario, come mostrato nel lemma \ref{lem:unitariety},
 \begin{equation}
 U_a\left(\ket{j} \right) := \ket{aj \mod p}.
\end{equation}
Svolgendo i calcoli, in quanto $a$ ha ordine $r$, risulta chiaro che 
\begin{equation}
  U_a\left(\ket{u_k} \right) = \exp(\frac{2 \pi i}{r} k) \ket{u_k},
\end{equation}
ovvero che i $k$ elementi $\ket{u_k}$ sono autovettori dell'operatore $U_a$, con autovalori $\exp(-\frac{2 \pi i}{r} k)$.
Operando, ora, come nella sezione \ref{sec:alg_ordine}, sull'algoritmo di individuazione dell'ordine, si dimostra che
\begin{equation}\label{eqn:sum_eigenvectors}
  \frac{1}{\sqrt{r}}\sum_{k=0}^{r-1}\ket{u_k} = \ket{1}.
\end{equation}
In particolare si può osservare che, applicando $x + sy$ volte l'operatore $U_a$, si manda $\ket{1}$ in $\ket{a^{x+sy} \mod p}$.
Inoltre lo stato $\ket{1}$ è descrivibile dalla somma in \eqref{eqn:sum_eigenvectors}, ovvero dalla somma di autovettori di $U_a$.
A partire da queste indicazioni si può facilmente desumere l'uguaglianza
\begin{equation}
 \ket{x,y,a^{x+sy} \mod p} = \ket{x,y} \frac{1}{\sqrt{r}}\sum_{k=0}^{r-1} \exp(\frac{2 \pi i}{r} k(x+sy)) \ket{u_k}.
\end{equation}
Segue che lo stato $\ket{\psi_2}$ può essere riscritto come
\begin{equation}
 \ket{\psi_2} = \frac{1}{2^t}\sum_{x,y = 0}^{2^t-1} \ket{x,y} \frac{1}{\sqrt{r}}\sum_{k=0}^{r-1} \exp(\frac{2 \pi i}{r} k(x+sy)) \ket{u_k}.
\end{equation}
Applicando la trasformata di Fourier inversa, infine, si arriva allo stato
\begin{equation}\label{eqn:DLP_last_state}
 \ket{\psi_3} = \frac{1}{\sqrt{r}}\sum_{k=0}^{r-1} \bigg(
  \sum_{x',y' = 0}^{2^t-1}
  \frac{1}{2^t}\big(\sum_{x = 0}^{2^t-1} \omega_{2^t}^{-xx'}\omega_{r}^{kx} \big) \ket{x'}
  \frac{1}{2^t}\big(\sum_{y = 0}^{2^t-1} \omega_{2^t}^{-yy'}\omega_{r}^{ksy}\big) \ket{y'}
 \bigg) \ket{u_k},
\end{equation}
in cui $\omega_r := \exp(\frac{2 \pi i}{r})$. Inoltre i coefficienti di $\ket{x'}$ sono:
\begin{equation}
 \alpha_{x'} :=
 \sum_{x = 0}^{2^t-1} \omega_{2^t}^{-xx'}\omega_{r}^{kx} = 
 \sum_{x = 0}^{2^t-1} \big(\omega_{2^t}^{-x'}\omega_{r}^{k} \big)^x =
 \sum_{x = 0}^{2^t-1} \big(e^{2 \pi i (k/r - x'/2^t)} \big)^x.
\end{equation}
Calcolando la somma parziale della serie geometrica, come in \eqref{eqn:dirichlet_kernel}, si ottiene
\begin{equation}
 \alpha_{x'} =
 \sum_{x = 0}^{2^t-1} \omega_{2^t}^{-xx'}\omega_{r}^{kx} = 
 \frac{1 - e^{2 \pi i 2^t(k/r - x'/2^t)}}{1 - e^{2 \pi i (k/r - x'/2^t)}}.
\end{equation}
Si può svolgere un calcolo analogo per i coefficienti di $\ket{y'}$ ed ottenere
\begin{equation}
 \alpha_{y'} :=
 \sum_{y = 0}^{2^t-1} \omega_{2^t}^{-yy'}\omega_{r}^{ksy} = 
 \frac{1 - e^{2 \pi i 2^t(ks/r - y'/2^t)}}{1 - e^{2 \pi i (ks/r - y'/2^t)}}.
\end{equation}
Sia, ora, $k$ fissato. 
Per studiare il funzionamento dell'algoritmo si riprendono le stime ricavate nello studio dell'algoritmo per la stima di fase, alla sezione \ref{sec:alg_phase}
Si introducono due valori $c$ e $d$, i quali rappresentano le migliori stime che possono essere misurate, rispettivamente per $2^tk/r$ e $2^tks/r$.
In particolare si pongono $c := \lfloor 2^tk/r \rfloor$ e $d := \lfloor 2^tks/r \rfloor$.
Si introduce anche una tolleranza $e$ su tali stime, che si ripercuote in una tolleranza $2^{-t}e$ sulla stima di $k/r$ e $ks/r$.
Chiamate $\varphi$ e $\psi$ le misure, rispettivamente del primo e del secondo registro, si ottengono le seguenti stime, ottenute come mostrato fino a \eqref{eqn:QPF_prob_err}:
\begin{align}
 \P\left(|\varphi - c| > e \right) &\leq \frac{1}{2(e-1)} \quad \text{ e }\\
 \P\left(|\psi - d| > e \right) &\leq \frac{1}{2(e-1)}.
\end{align}
Seguendo sempre l'analisi per l'algortimo di stima di fase si ottiene un'espressione per la lunghezza $t$ dei primi due registri.
Si vuole ottenere una stima di $k/r$ e $ks/r$ accurata a $2L + 1$ bit, con probabilità almeno $1 - \epsilon$.
Si deve, quindi, scegliere $t = 2L + 1 + \left\lceil \log_2\left(2 + \frac{1}{2\epsilon}\right)\right\rceil$.

Inoltre, lo stato \eqref{eqn:DLP_last_state} è, nel terzo registro, in una sovrapposizione uniforme degli autovettori $\ket{u_k}$ di $U_a$.
La probabilità di misurare una stima accurata, per un fissato $k$, di $k/r$ e $ks/r$ sarà di $(1-\epsilon)/k$.

La scelta di prendere $t$ maggiore di $2L+1$ è giustificata dalla necessità di operare, come per l'individuazione dell'ordine, descritto in sezione \ref{sec:alg_ordine}, con l'algoritmo dell'espansione in frazione continua.
Tale scelta, infatti, permette di rispettare le ipotesi del teorema \ref{thm:Cont_frac}, il quale garantisce che $k/r$ è un convergente della frazione continua di $\varphi$ e $ks/r$ di $\psi$.

Operando con l'algoritmo per le frazioni continue, quindi, si ottiengono una frazione ridotta ai minimi termini per $k/r$ e una per $ks/r$.
Moltiplicando i due risultati per $r$, che è un valore noto, si ottengono sempre $k$ e $ks$.
Dividendo la seconda stima per la prima si ottiene esattamente $s$, come richiesto.
La probabilità che tale procedura abbia successo dipende solamente dall'accuratezza della stima di $k/r$ e di $ks/r$.
Esse hanno, indipendentemente, probabilità $1-\epsilon$ di rientrare nei criteri stabiliti, ovvero l'algoritmo produce un risultato corretto con probabilità almeno $\left(1 - \epsilon \right)^2$.

Si può, inoltre, generalizzare questo algoritmo a ogni gruppo moltiplicativo.
Per far ciò $a$ può essere scelto in modo arbitrario, tra gli elementi di ordine maggiore di $1$.
$b$, invece, deve essere una potenza di $a$, per cui deve essere scelto in $\langle a \rangle$.
Grazie all'algoritmo di ricerca dell'ordine si può individuare l'ordine $r$ di $a$ e operare con quel valore, al posto che $p-1$, implicitamente considerato in questa sezione.

\subsubsection{Costo computazionale}
Percorrendo il circuito in ordine di esecuzione si incontrano:
Due gate di Hadamard, con un costo computazionale di $O(L)$ ciascuno, per un totale di $O(L)$.
Il gate $U_f$, la cui implementazione può essere data da due algoritmi per l'esponenziazione modulare in serie.
Ciascuno di essi ha un costo di $O(L^3)$ operazioni, come mostrato in \ref{algo:modular_exponentiation}, nella sezione per l'algoritmo di ricerca dell'ordine.
Il gate $U_f$ ha, quindi, un costo complessivo di $O(L^3)$.
Compaiono, infine, due trasformate di Fourier inverse, che richiedono $O(L^2)$ operazioni elementari ciascuna.

Segue a questo l'esecuzione dell'algoritmo per le frazioni continue il quale ha un costo di $O(L^3)$ operazioni.
In realtà questo algoritmo può essere eseguito da un computer classico, lasciando libero il computer quantistico di operare con algoritmi più consoni alle proprie caratteristiche.

Combinando tutti questi risultati si ottiene che la complessità della dimensione del circuito  per la risoluzione del problema del logaritmo discreto è di $O(L^3)$ porte logiche elementari.

\begin{algo}[Logaritmo discreto quantistico]\*
\begin{description}
  \item [Input:] (1) $a,b \in \left(\Zn{p}\right)^*$ t.c. $\left(\Zn{p}\right)^* = \langle a \rangle$, (2) $2$ registri contenenti $t = 2L + 1 + {\left\lceil \log_2 \left(2 + \frac{1}{2\epsilon} \right) \right\rceil}$ qubit inizializzati nello stato $\ket{0}$, (3) $L$ qubit inizializzati nello stato $\ket{0}$.
  \item [Output:] il minimo $s \in \N$ tale che $a^s = b \mod p$.
  \item [Runtime:] $O(L^3)$ operazioni. Ritorna un output corretto con probabilità $\left(1-\epsilon\right)^2$.
  \newpage
  \item [Procedura:] \ 
    \begin{enumerate}
     \item ${\displaystyle\ket{0}\ket{0}\ket{0}}$
     \hfill {\small stato iniziale}
     \item ${\displaystyle\stackrel{H^{\otimes 2t}}{\longmapsto} \frac{1}{2^t} \sum_{x,y=0}^{2^t-1} \ket{x}\ket{y}\ket{0}}$
     \hfill {\small sovrapposizione iniziale}
     \item ${\displaystyle\stackrel{U_a}{\longmapsto} \frac{1}{2^t} \sum_{x,y = 0}^{2^t-1} \ket{x,y,a^{x+sy} \mod p}}$
     \hfill {\small risultato $U_a$}\\*
     \ \ ${\displaystyle\quad = \frac{1}{\sqrt{r}}\sum_{k=0}^{r-1} \bigg(\sum_{x',y' = 0}^{2^t-1}
     \frac{1}{2^t}\big(\sum_{x = 0}^{2^t-1} \omega_{2^t}^{-xx'}\omega_{r}^{kx} \big) \ket{x'}
     \frac{1}{2^t}\big(\sum_{y = 0}^{2^t-1} \omega_{2^t}^{-yy'}\omega_{r}^{ksy}\big) \ket{y'}\bigg) \ket{u_k}}$
     \item ${\displaystyle\stackrel{FT^\dag}{\longmapsto} \frac{1}{\sqrt{r}}\sum_{s=0}^{p-2} \widetilde{\ket{k/r}}\widetilde{\ket{ks/r}}\ket{u_k}}$
     \hfill {\small trasformata Fourier inversa}
     \item ${\displaystyle\longmapsto \left(\widetilde{k/r},\widetilde{ks/r}\right)}$
     \hfill {\small misura primi due registri}
     \item ${\displaystyle\longmapsto \left(k,ks\right)}$
     \hfill {\small algoritmo frazioni continue}
     \item ${\displaystyle\longmapsto s}$
    \end{enumerate}
 \end{description}
\end{algo}

\begin{abstract}
Negli ultimi decenni si sono registrati importanti sviluppi nel campo dell'informatica quantistica, branca di ricerca nata a partire dall'idea di Feynman di sfruttare sistemi quantistici per eseguire computazioni.
Recentemente, grazie alla grande ricerca tecnologica, si è vista la realizzazione dei primi computer quantistici.
Parallelamente la ricerca teorica nel campo della computazione quantistica ha portato allo sviluppo di svariati algoritmi con richieste computazionali minori rispetto ai corrispettivi classici, in alcuni casi addirittura in maniera esponenziale.

Alcuni algoritmi quantistici riescono a risolvere in modo efficiente -- ovvero con una richiesta di operazioni che dipende in modo polinomiale dalla dimensione del dato in input -- problemi considerati ``difficili'' -- ovvero non risolubili in modo efficiente -- nell'ambito della computazione classica.
Questo, dal punto di vista pratico, potrebbe minare la sicurezza delle più diffuse implementazioni di sistemi di crittografia a chiave pubblica.
Dal punto di vista teorico, invece, mette in dubbio la veridicità della tesi forte di Church-Turing, la quale afferma che, per ogni problema risolubile in modo efficiente su un arbitrario sistema computazionale, si possa trovare un algoritmo classico efficiente.
Risultati in questa direzione porterebbero ad approfondire le conoscenze necessarie per affrontare il problema del millennio \textbf{P} vs \textbf{NP}, potenzialmente accentuando l'importanza del paradigma quantistico nella teoria della complessità.

L'elaborato, traendo spunto da \cite{Book:QCQI} per la parte propriamente di computazione quantistica e da \cite{Book:CCModern} e \cite{Book:PapadimitriouCC} per la parte riguardante la complessità computazionale, ha l'obiettivo di spiegare il funzionamento di alcuni noti algoritmi quantistici, dando gli strumenti necessari a comprenderne l'importanza dal punto di vista della teoria della complessità computazionale.

Inizia, nel primo capitolo, introducendo le basi di algebra lineare e di meccanica quantistica necessarie per poter definire circuiti e algoritmi quantistici.
Vengono, inoltre, descritti il funzionamento di un computer quantistico e alcune differenze del nuovo paradigma di computazione da quello classico, in particolare mostrando operazioni non implementabili su un substrato quantistico.

Nel secondo capitolo si introducono gli strumenti alla base della complessità computazionale, che saranno utilizzati per analizzare le richieste degli algoritmi quantistici e i vantaggi rispetto alle controparti classiche.
A tal fine viene introdotto il concetto di classe di complessità e ne vengono dati esempi, tra cui le celeberrime \textbf{P} ed \textbf{NP}, per dare un contesto alla teoria citata.

Nel terzo capitolo si analizzano i primi algoritmi, inizialmente mostrando  semplici applicazioni del comportamento quantistico dell'hardware, tra cui il parallelismo e il teletrasporto quantistici, che stanno alla base dei grandi vantaggi computazionali rispetto alla computazione classica.
Il lavoro finisce con la descrizione del funzionamento di una lista di algoritmi che sfruttano le peculiarità del nuovo paradigma computazionale per ottenere una riduzione nelle richieste computazionali, a volte esponenziale, rispetto alle più efficienti controparti classiche.
Tra questi figurano gli importantissimi algoritmi per l'individuazione dell'ordine, per la fattorizzazione degli interi e per il calcolo del logaritmo discreto, descritti per la prima volta da Peter Shor in \cite{Article:Shor:1997}.
\end{abstract}

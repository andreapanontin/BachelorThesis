\chapter{Frazioni continue}\label{app:Cont_frac}
\begin{defn}[frazione continua finita]
 Si definisce {\upshape frazione continua finita} la funzione
 \begin{equation}
  [a_0;a_1,a_2,\dots,a_n] :=
  a_0 + \cfrac{1}{a_1 
      + \cfrac{1}{a_2 
      + \cfrac{1}{\dots + \cfrac{1}{a_n} } } }
 \end{equation}
 delle $n + 1$ variabili $a_0, a_1, \dots, a_n \in \R$.
 
 Si chiama {\upshape $j$-esimo convergente} della frazione continua il termine
 \begin{equation}
  [a_0;a_1,a_2,\dots,a_j] = 
  a_0 + \cfrac{1}{a_1 
      + \cfrac{1}{a_2 
      + \cfrac{1}{\dots + \cfrac{1}{a_j} } } },
 \end{equation}
 dove $0 \leq j < n$.
\end{defn}

\begin{thm}\label{thm:Cont_frac_convergents}
 Sia $a_0,\dots, a_N$ una successione di reali positivi. Si ha che
 \begin{equation}
  [a_0;a_1,\dots,a_n] = \frac{p_n}{q_n}
 \end{equation}
 dove $p_n$ e $q_n$ sono numeri reali definiti iterativamente da:\\
 (1) $p_0 := a_0$ e $q_0:=1$\\
 (2) $p_1 := 1 + a_0a_1$ e $q_1:=a_1$\\
 (3) per $2 \leq n \leq N$
 \begin{align}
  p_n :=& a_np_{n-1} + p_{n-2}\\
  q_n :=& a_nq_{n-1} + q_{n-2}
 \end{align}
 Se, inoltre, $a_j \in \N^+$ per ogni $j$, segue che anche $p_j, q_j \in \N^+$ per ogni $j$. 
\end{thm}
\begin{proof}
 Nel caso di $n=0$ è ovvio, mentre nei casi $n=1,2$ è una semplice verifica.
 Si opera, poi, per induzione su $n$, con caso base $n=2$, già affrontato.
 
 Sia, ora $n \geq 3$, e valga l'ipotesi induttiva.
 Risulta chiaro, per definizione di frazione continua, che vale la seguente
 \begin{equation}\label{eqn:app_Induz_1}
  [a_0;\dots, a_{n-1}, a_n] = [a_0;\dots,a_{n-2},a_{n-1} + \frac{1}{a_n}]
 \end{equation}
 Siano $\tilde{p}_j$ e $\tilde{q}_j$ le successioni, definite come nell'enunciato del teorema, associate alla frazione continua che compare nel membro a destra di \eqref{eqn:app_Induz_1}.
 Chiaramente $\tilde{p}_{n-3} = p_{n-3}$, $\tilde{p}_{n-2} = p_{n-2}$ e $\tilde{q}_{n-3} = q_{n-3}$, $\tilde{q}_{n-2} = q_{n-2}$, dove $p_j$ e $q_j$ sono le successioni associate alla frazione continua a sinistra in \eqref{eqn:app_Induz_1}.
 
 Per ipotesi induttiva vale la seguente uguaglianza:
 \begin{equation}\label{eqn:app_Induz_2}
  [a_0;\dots,a_{n-2},a_{n-1} + \frac{1}{a_n}] = \frac{\tilde{p}_{n-1}}{\tilde{q}_{n-1}}.
 \end{equation}
 Dove, per definizione di $\tilde{p}_{n-1}$ e $\tilde{q}_{n-1}$, la frazione a destra può essere riscritta come
 \begin{align}
  \frac{\tilde{p}_{n-1}}{\tilde{q}_{n-1}} &= \frac{(a_{n-1} + 1/a_n)p_{n-2} + p_{n-3}}{(a_{n-1} + 1/a_n)q_{n-2} + q_{n-3}}\\
  &= \frac{p_{n-1} + p_{n-2}/a_n}{q_{n-1} + q_{n-2}/a_n},
 \end{align}
 in cui, l'ultima uguaglianza è dovuta alla definizione di $p_{n-1}$ e $q_{n-1}$.
 Moltiplicando l'ultimo termine per $1 = a_n/a_n$ si ottiente
 \begin{equation}\label{eqn:app_Induz_3}
  \frac{\tilde{p}_{n-1}}{\tilde{q}_{n-1}} = \frac{a_np_{n-1} + p_{n-2}}{a_nq_{n-1} + q_{n-2}} = \frac{p_n}{q_n}
 \end{equation}
 Combinando le equazioni \eqref{eqn:app_Induz_1}, \eqref{eqn:app_Induz_2} e \eqref{eqn:app_Induz_3} si ottiene
 \begin{equation}
 [a_0;\dots, a_{n-1}, a_n] = \frac{p_n}{q_n}
 \end{equation}
 ovvero la tesi.
\end{proof}
Si ha, quindi, un algoritmo per trovare le frazioni associate ai successivi convergenti di una data frazione continua. Inoltre vale il seguente lemma:
\begin{lem}\label{lem:Cont_frac_coprimality}
 $q_np_{n-1} - p_nq_{n-1} = (-1)^n$ p ogni $n \geq 1$
\end{lem}
\begin{proof}
 Si opera per induzione su $n$.
 Sia $n=1$, allora 
 \begin{equation}
  q_1p_0 - p_1q_0 = a_1a_0 - (1 + a_0a_1) = -1 = (-1)^n.
 \end{equation}
 Sia $n > 1$. Per definizione iterativa di $p_n$ e $q_n$ vale
 \begin{align}
  q_np_{n-1} - p_nq_{n-1} &= (a_nq_{n-1} + q_{n-2})p_{n-1} - (a_np_{n-1} + p_{n-2})q_{n-1}\\
  &=  q_{n-2}p_{n-1} - p_{n-2}q_{n-1} = (-1)(q_{n-1}p_{n-2} - p_{n-1}q_{n-2})
 \end{align}
 Per ipotesi induttiva $q_{n-1}p_{n-2} - p_{n-1}q_{n-2} = (-1)^{n-1}$, da cui si ottiene
 \begin{equation}
  q_np_{n-1} - p_nq_{n-1} = (-1)(-1)^{n-1} = (-1)^n
 \end{equation}
\end{proof}

Non era necessario restringere ulteriormente la definizione di \textit{frazione continua} per ottenere i risultati precedenti.
Il concetto con cui opereremo, però, è il più specializzato concetto di frazione continua semplice, di cui segue la definizione.

\begin{defn}[frazione continua finita semplice]
 Si definisce {\upshape frazione continua finita semplice} una frazione continua finita 
 \begin{equation}
  [a_0;a_1,a_2,\dots,a_n] := 
  a_0 + \cfrac{1}{a_1 
      + \cfrac{1}{a_2 
      + \cfrac{1}{\dots + \cfrac{1}{a_n} } } },
 \end{equation}
 in cui $a_0, \dots, a_n \in \N^+ := \left\{1, 2, \dots \right\}$.
\end{defn}
Risulta chiaro che ogni frazione continua finita semplice rappresenta un numero razionale, però, per i numeri razionali maggiori di 1, vale anche il viceversa. È utile introdurre un potente ed efficiente algoritmo per il calcolo delle frazioni continue per mostrare tale risultato:
\begin{algo}[Espansione in frazione continua]\ \\
 Ogni numero razionale maggiore di 1 ha una rappresentazione in frazione continua finita semplice, ovvero con un'espressione come
 \begin{equation}
  [a_0;a_1,a_2,\dots,a_n] := 
  a_0 + \cfrac{1}{a_1 
      + \cfrac{1}{a_2 
      + \cfrac{1}{\dots + \cfrac{1}{a_n} } } },
 \end{equation}
 in cui $a_0, \dots, a_n \in \N^+$.
 Questa espansione si può trovare operando con il seguente
 \begin{description}
  \item[Algoritmo:] Sia $x = p_0/p_1$ un numero razionale. Si ricavano, iterativamente, $a_j$ operando nel seguente modo. $a_0 = \lfloor x \rfloor$ è la parte intera di $x$, per cui 
  \begin{equation}
   x = a_0 + \frac{p_2}{p_1} = a_0 +\frac{1}{\frac{p_1}{p_2}},
  \end{equation}
  in cui, nell'ultimo passaggio, si è invertita la frazione. In particolare, in quanto $a_0 = \lfloor x \rfloor$ segue che $p_2/p_1 < 1$, ovvero $p_2 < p_1$.
  Chiamato $x_1 := p_1/p_2$ si ripete su $x_1$ il processo appena svolto, ad ottenere $a_1$ e $x_2 := p_2/p_3$, con $p_3 < p_2$.
  Iterando si calcolano $a_n$, $x_{n+1}$ e $p_{n+2}$ a partire da $x_n$, costruendo un'espansione in frazione continua per $x$.
  Tale algoritmo si arresta quando $x_n = \lfloor x_n \rfloor$ è intero, ovvero quando $p_{n+1} = 0$.
  In particolare, per ogni $x$ razionale, $p_j$ è una successione a coefficienti in $\N^+$ e strettamente decrescente. Ne consegue che, in un numero finito di passi $p_j = 0$, ovvero l'algoritmo si ferma.
 \end{description}
\end{algo}
In realtà, oltre all'espansione in frazione continua semplice trovata da questo algoritmo, per ogni numero razionale si può trovare un'espansione equivalente. Per fare ciò, all'ultimo passo, invece che scegliere $a_n = a_n$, si può esprimere $a_n = (a_n - 1) + 1/1$, ottenendo un'espansione con $\tilde{a}_n = a_n - 1$ e $\tilde{a}_{n+1} = 1$. Quest'ambiguità tornerà utile, in quanto permette di esprimere un numero razionale con una frazione continua semplice con numero pari o dispari di coefficienti.
Inoltre tale algoritmo, se si concede $a_0 = 0$, può individuare espansioni in frazioni continue ``semplici'' per numeri razionali positivi, anche minori di 1.

Prima di passare al prossimo risultato è utile evidenziare un paio di osservazioni.
Innanzitutto, nel caso di frazioni continue semplici, il lemma \ref{lem:Cont_frac_coprimality} ha come conseguenza che, per ogni $n$, $p_n$ e $q_n$ sono coprimi tra loro. 
Questo segue dall'identità di Bezout, la quale afferma che, per $m,n \in \N$, esistono $a.b \in \Z$ tali che $am + bn =$MCD$(a,b)$.
Se $q_np_{n-1} - p_nq_{n-1} = (-1)^n$ segue che MCD$(q_n,p_n) \mid 1$, ovvero la coprimalità.
Si ha, quindi, che, nel calcolo dei convergenti, l'algoritmo dato dal teorema \ref{thm:Cont_frac_convergents} restituisce solo frazioni ridotte ai minimi termini.

Un'ulteriore conseguenza utile del teorema \ref{thm:Cont_frac_convergents} segue dalla definizione iterativa di $p_n$ e di $q_n$. Nel caso di una frazione continua semplice $a_0, \dots, a_n \in \N^+$, per cui $\nexists j : a_j = 0$, implica che 
\begin{equation}\label{eqn:Cont_frac_convergents_increasing}
 p_n = a_np_{n-1} + p_{n-2} = a_n(a_{n-1}p_{n-2} + p_{n-3}) + p_{n-2} > 2p_{n-2}.
\end{equation}

Per concludere questa sezione, infine, è necessario dimostrare un risultato fondamentale nell'implementazione dell'algoritmo per l'individuazione dell'ordine, ovvero il seguente teorema:
\begin{thm}
 Siano $x$ ed $p/q$ razionali tali che
 \begin{equation}\label{eqn:HP_thm_Convergent}
  \left| \frac{p}{q} - x \right| \leq \frac{1}{2q^2}
 \end{equation}
 Allora $p/q$ è un convergente della frazione continua di $x$
\end{thm}
\begin{proof}
 Se, per caso $x = p/q$, l'affermazione è ovvia.
 Altrimenti sia $p/q = [a_0;\dots, a_n]$ l'espansione in frazione continua semplice di $p/q$.
 Siano $p_j$ e $q_j$ definiti come specificato nel teorema \ref{thm:Cont_frac_convergents}, in modo tale che $p_n/q_n = p/q$.
 Si definisce, inoltre, $\delta$ in modo che
 \begin{equation}
  x = \frac{p_n}{q_n} + \frac{\delta}{2q_n^2}.
 \end{equation}
 Da \eqref{eqn:HP_thm_Convergent} e dal fatto che $q_n \mid q$ si ha che $0 < |\delta| < 1$.
 Si definisce, infine, $\lambda$ come dalla seguente equazione
 \begin{equation}
  \lambda := 2 \left( \frac{q_np_{n-1} - p_nq_{n-1}}{\delta} \right) - \frac{q_{n-1}}{q_n}.
 \end{equation}
 Questa espressione è scelta affinché $\lambda$ soddisfi l'equazione
 \begin{equation}
  x = \frac{\lambda p_n + p_{n-1}}{\lambda q_n + q_{n-1}},
 \end{equation}
 da cui una frazione continua per $x$ è $[a_0;\dots,a_n,\lambda]$.
 Si sceglie, ora, $n$ pari se $\delta > 0$ e $n$ dispari se $\delta < 0$, come concesso dall'osservazione che segue l'algoritmo per l'espansione in frazione continua. Per il lemma \ref{lem:Cont_frac_coprimality}, $q_np_{n-1} - p_nq_{n-1} = (-1)^n$. Segue che 
 \begin{equation}
  2 \left( \frac{q_np_{n-1} - p_nq_{n-1}}{\delta} \right) = \frac{2}{|\delta|} > 0,
 \end{equation}
 e, in particolare, $\lambda$ diventa
 \begin{equation}
  \lambda = \frac{2}{\delta} - \frac{q_{n-1}}{q_n}.
 \end{equation}
 In quanto $|\delta|<1$ e la successione $q_j$ è strettamente crescente vale che
 \begin{equation}
  \lambda = \frac{2}{\delta} - \frac{q_{n-1}}{q_n} > 2 - 1 > 1
 \end{equation}
 ovvero $\lambda$ è un numero razionale e maggiore di 1.
 Per questo motivo ammette un'espansione in frazione continua finita e semplice $\lambda = [b_0, \dots, b_m]$.
 Segue da ciò che l'espansione \textit{semplice} per $x$ è $[a_0, \dots, a_n, b_0, \dots, b_m]$, di cui $p/q$ è un convergente. 
\end{proof}
